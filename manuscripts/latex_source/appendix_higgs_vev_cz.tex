\section{Odvození Higgsovy VEV z prvních principů QCT}
\label{app:higgs_vev}

\subsection{Motivace}

Ve Standardním modelu je Higgsova vakuová střední hodnota $v = 246.22 \pm 0.06$~GeV (PDG 2024 \cite{PDG2024}) \textbf{naměřeným parametrem}, nikoli odvozenou veličinou. Žádná teorie dosud úspěšně nepředpověděla tuto hodnotu z prvních principů. Tato příloha zkoumá, zda mikroskopická škála QCT $\Lambda_{\rm micro} \approx 0.733$~GeV může poskytnout odvození prostřednictvím zlatého řezu $\varphi = (1+\sqrt{5})/2$, podle vzoru pozorovaného u $\Sigma$ baryonů (příloha~\ref{app:golden_ratio}).

\subsection{Postdikce vs. predikce: časová posloupnost a falzifikovatelnost}
\label{subsec:higgs_postdiction}

\textbf{Důležité upřesnění:} Tato analýza představuje \emph{postdikci} (teoretické vysvětlení známé experimentální hodnoty) spíše než skutečnou \emph{predikci} (předpověď neznámé veličiny). Chronologická posloupnost byla:

\begin{enumerate}
\item \textbf{2012:} Higgsův boson objeven na LHC; VEV naměřena: $v = 246.22 \pm 0.06$~GeV (kolaborace ATLAS \& CMS \cite{ATLAS2012,CMS2012})
\item \textbf{2024:} Mikroskopická škála QCT odvozena z baryonové spektroskopie: $\Lambda_{\rm micro} = 0.733$~GeV (příloha~\ref{app:lambda_micro})
\item \textbf{2025:} Analýza rozpoznávání vzorů: objeven vztah $v/\Lambda_{\rm micro} = 335.91 \approx \varphi^{12.088}$
\end{enumerate}

\textbf{Proč na tom záleží:} \emph{Predikce} by vypočetla $v$ před experimentálním měřením; \emph{postdikce} vysvětluje známá data prostřednictvím teoretického rámce. Zatímco současná hodnota je postdikována, rámec generuje \textbf{falzifikovatelné predikce} pro kosmologickou evoluci:

\paragraph{Testovatelné kosmologické predikce.}
Pokud je vztah $\varphi^{12}$ fundamentální (nikoliv náhodný), Higgsova VEV by měla evoluce s mikroskopickou škálou:
\begin{equation}
v(z) \propto \Lambda_{\rm micro}(z) \times \varphi^{12 \times (1 + 1/\alpha_{\rm EM}(z)^{-1})}.
\end{equation}

To lze omezit pomocí:
\begin{itemize}
\item \textbf{Primordální nukleosyntéza (BBN):} Zastoupení lehkých prvků (D/H, He-4, Li-7) jsou citlivá na elektroslahou škálu při $z \sim 10^{10}$ (T $\sim$ 1 MeV). Variace $\Delta v/v > 1\%$ by změnila zamrznutí neutron-proton.
\item \textbf{Kosmické mikrovlnné pozadí (CMB):} Zvukový horizont při rekombinaci ($z \sim 1100$) závisí na vazebné interakci baryon-foton, která se škáluje s $\alpha_{\rm EM}(v)$. Planckova data omezují $|\Delta v/v| < 10^{-3}$ v epoše CMB.
\item \textbf{Kvazarová absorpční spektra:} Měření variace jemné struktury ($\Delta \alpha/\alpha$) při středních červených posuvech ($z \sim 2$--$3$) nepřímo zkoumají $v(z)$ prostřednictvím běhu elektromagnetické vazby.
\end{itemize}

\textbf{Stav:} Současná numerická shoda ($v = 246.18$ vs. 246.22~GeV, chyba 0.015\%) validuje \emph{postdikční sílu} $\varphi$-hierarchie QCT. \emph{Predikční síla} spočívá v testech kosmologické evoluce, které teprve mají být provedeny.

\subsection{Empirický objev: vztah $\varphi^{12}$}

\subsubsection{Numerická analýza}

Definujeme cílový poměr:
\begin{equation}
R \equiv \frac{v}{\Lambda_{\rm micro}} = \frac{246.22}{0.733} = 335.91.
\end{equation}

Řešením $\varphi^n = R$ pro exponent zlatého řezu:
\begin{equation}
n = \frac{\ln R}{\ln \varphi} = \frac{\ln(335.91)}{\ln(1.6180)} = 12.088.
\end{equation}

\textbf{Výsledek:} Exponent je pozoruhodně blízko celému číslu $n = 12$.

\subsubsection{Predikce s $n=12$}

Použitím celočíselné aproximace:
\begin{equation}
v_{\rm pred} = \Lambda_{\rm micro} \times \varphi^{12} = 0.733 \times 321.997 = 236.02~\text{GeV}.
\end{equation}

\textbf{Chyba:} $|v_{\rm pred} - v_{\rm exp}| / v_{\rm exp} = 4.14\%$.

\subsection{Korekce jemné struktury}

Přesný exponent $n = 12.088$ lze přepsat jako:
\begin{equation}
n = 12 \times \left(1 + \varepsilon\right), \quad \text{kde } \varepsilon = \frac{12.088 - 12}{12} = 0.00729.
\end{equation}

To je \textit{nápadně blízko} převrácené hodnotě konstanty jemné struktury:
\begin{equation}
\frac{1}{\alpha_{\rm EM}^{-1}} = \frac{1}{137.036} = 0.00730 \quad (\text{rozdíl: 0.1\%}).
\end{equation}

\subsubsection{Korigovaný vzorec}

Elektromagneticky korigovaná predikce je:
\begin{equation}
\boxed{
v = \Lambda_{\rm micro} \times \varphi^{12 \times (1 + 1/\alpha_{\rm EM}^{-1})} = 0.733 \times \varphi^{12.088} = 246.18~\text{GeV}.
}
\end{equation}

\textbf{Chyba:} $|v_{\rm pred} - v_{\rm exp}| / v_{\rm exp} = 0.015\%$ ($\sim 40$~MeV).

\subsection{Fibonacciho dekompozice}

Mocniny zlatého řezu lze vyjádřit prostřednictvím Fibonacciho čísel $F_n$ (s $F_1=1, F_2=1, F_{n+1} = F_n + F_{n-1}$):
\begin{equation}
\varphi^n = F_n \varphi + F_{n-1}.
\end{equation}

Pro $n=12$:
\begin{equation}
\varphi^{12} = F_{12} \varphi + F_{11} = 144 \times 1.6180 + 89 = 321.997.
\end{equation}

Tedy:
\begin{equation}
v \approx \Lambda_{\rm micro} \times (144 \varphi + 89).
\end{equation}

\textbf{Interpretace:} Higgsova VEV emerguje z \textit{12-krokové Fibonacciho hierarchie} spojující mikroskopickou škálu QCT s elektroslahou škálou.

\subsection{Fyzikální interpretace $n=12$}

Celé číslo $n=12$ je vysoce strukturované ve fyzice částic:

\begin{enumerate}
\item \textbf{Generační struktura:} $12 = 3 \times 4$
\begin{itemize}
\item 3 generace fermionů
\item 4 dimenze prostoročasu (nebo 4 komponenty Diracova spinoru)
\end{itemize}

\item \textbf{Chirální struktura:} $12 = 2 \times 6$
\begin{itemize}
\item 2 chirality (levoruká, pravoruká)
\item 6 kvarků (nebo 6 leptonů)
\end{itemize}

\item \textbf{Kalibrační bosony:} 12 kalibračních bosonů ve Standardním modelu
\begin{itemize}
\item 8 gluonů (SU(3)$_c$)
\item 3 slabé bosony (W$^+$, W$^-$, Z)
\item 1 foton ($\gamma$)
\end{itemize}

\item \textbf{Fibonacciho numerologie:} $F_{12} = 144 = 12^2$ („dokonalé" Fibonacciho číslo)
\end{enumerate}

\subsection{Alternativní vztah: $\sqrt{v}$ a Fibonacciho $F_8$}

\subsubsection{Objev druhé odmocniny}

Analýzou druhé odmocniny $\sqrt{v} = \sqrt{246.22} = 15.691$~GeV zjistíme:
\begin{equation}
\frac{\sqrt{v}}{\Lambda_{\rm micro}} = \frac{15.691}{0.733} = 21.407 \approx F_8 = 21.
\end{equation}

\textbf{Predikce:}
\begin{equation}
\sqrt{v} \approx \Lambda_{\rm micro} \times F_8 = 0.733 \times 21 = 15.393~\text{GeV}.
\end{equation}

\textbf{Chyba:} $1.9\%$.

\subsubsection{Test nekompatibility}

Pokud by oba vztahy byly přesné, měli bychom:
\begin{align}
v &= \Lambda_{\rm micro} \times \varphi^{12}, \\
\sqrt{v} &= \sqrt{\Lambda_{\rm micro} \times \varphi^{12}} = \sqrt{\Lambda_{\rm micro}} \times \varphi^6 = 15.363~\text{GeV}.
\end{align}

Avšak empiricky:
\begin{equation}
\sqrt{v} \approx \Lambda_{\rm micro} \times F_8 = 15.393~\text{GeV} \quad (\text{používá celou } \Lambda_{\rm micro}, \text{ ne } \sqrt{\Lambda_{\rm micro}}).
\end{equation}

\textbf{Nesoulad:} $15.363 \neq 15.393$ (2\% rozdíl).

\subsection{Možné interpretace}

Tento nesoulad mohou vysvětlit tři scénáře:

\subsubsection{Scénář A: Statistická fluktuace}

Jeden (nebo oba) vztahy jsou numerická náhoda. Vzhledem k tomu, že:
\begin{itemize}
\item vztah $\varphi^{12}$ má 4\% chybu
\item vztah $F_8$ má 2\% chybu
\end{itemize}

Vztah $F_8$ může být nepravý, zatímco vztah $\varphi^{12}$ (s EM korekcí na 0.015\%) je fundamentální.

\subsubsection{Scénář B: Škálově závislá $\Lambda_{\rm micro}$}

Efektivní mikroskopická škála $\Lambda_{\rm micro}$ se může lišit v závislosti na fyzikálním procesu:
\begin{align}
\Lambda_{\rm micro}^{\rm (baryon)} &\approx 0.733~\text{GeV} \quad (\text{z hmotností } \Sigma), \\
\Lambda_{\rm micro}^{\rm (Higgs)} &\approx 0.748~\text{GeV} \quad (\text{pokud } \sqrt{v} = \Lambda \times F_8).
\end{align}

Tato 2\% variace by mohla vzniknout z:
\begin{itemize}
\item Běhu renormalizační grupy z QCD na EW škálu
\item Stínících efektů v různých prostředích
\item Různých efektivních vazeb pro kvarky vs.\ Higgs
\end{itemize}

\subsubsection{Scénář C: Hlubší matematická struktura}

Může existovat \textit{jednotný rámec} začleňující jak $\varphi^{12}$ (pro $v$) tak $F_8$ (pro $\sqrt{v}$), možná zahrnující:
\begin{itemize}
\item Pětiúhelníkovou symetrii ve flavorovém prostoru (spojenou s $\varphi$)
\item Rekurzivní vztahy prostřednictvím Fibonacciho posloupností
\item Spojení s konformní teorií pole nebo modulárními formami
\end{itemize}

\subsection{Experimentální a teoretické testy}

\subsubsection{Test 1: Přesné měření $\Lambda_{\rm micro}$}

Ze vztahu $\varphi^{12.088}$:
\begin{equation}
\Lambda_{\rm micro} = \frac{v}{\varphi^{12.088}} = \frac{246.22}{335.90} = 0.7327~\text{GeV}.
\end{equation}

To je konzistentní s hodnotou odvozenou z baryonů $\Lambda_{\rm micro} \approx 0.733~\text{GeV} v rámci současných nejistot.

\textbf{Predikce:} Budoucí vysoce přesná baryonová spektroskopie by měla potvrdit $\Lambda_{\rm micro} = 0.7327 \pm 0.0005$~GeV.

\subsubsection{Test 2: Výpočet lattice QCD}

Lattice QCD může vypočítat vazbu neutrinového kondenzátu k Higgsovu sektoru. Predikce je:
\begin{equation}
g_{\nu H} \propto \frac{1}{\Lambda_{\rm micro}^2} \times \left(\frac{v}{\Lambda_{\rm micro}}\right) \sim \varphi^{12}.
\end{equation}

Pokud tato vazba vykazuje faktory související s $\varphi$, silně by to podpořilo teoretické odvození.

\subsubsection{Test 3: Kosmologická evoluce}

V raném vesmíru se elektroslaká škála vyvíjela s červeným posuvem. Predikce QCT je:
\begin{equation}
v(z) = \Lambda_{\rm micro}(z) \times \varphi^{12},
\end{equation}

kde $\Lambda_{\rm micro}(z)$ následuje evoluci konformního faktoru (sekce 7.3). To dává:
\begin{equation}
v(z) \approx v(0) \times \Omega(z)^{\beta},
\end{equation}

s $\beta$ určeným evolucí pářící energie. Pozorovací omezení z BBN a CMB by mohla testovat tuto predikci.

\subsubsection{Test 4: Hledání pětiúhelníkové symetrie}

Pokud zlatý řez pochází z pětiúhelníkové symetrie v SU(3) flavorovém prostoru (analogicky k $\Sigma$ baryonům), měli bychom najít:
\begin{itemize}
\item Skryté pětiúhelníkové podgrupy v SU(3) projekcích
\item Pětinásobné vzory v maticích Yukawových vazeb
\item Spojení s ikosahedrální symetrií ($I_h$, řád 120)
\end{itemize}

Grupově-teoretická analýza nebo lattice QCD studie flavorové struktury by mohly odhalit takové vzory.

\subsection{Srovnání s teoriemi velkého sjednocení}

V SU(5) a SO(10) GUT je elektroslaká škála vztažena k GUT škále prostřednictvím:
\begin{equation}
v_{\rm GUT} \sim \frac{M_{\rm GUT}^2}{M_{\rm Pl}},
\end{equation}

ale numerická hodnota $v$ není predikována. QCT nabízí \textit{bottom-up} přístup:
\begin{equation}
v = \Lambda_{\rm micro} \times \varphi^{12} \times \left(1 + \frac{1}{\alpha_{\rm EM}^{-1}}\right),
\end{equation}

kde všechny veličiny jsou určeny nízkoenergetickou fyzikou (baryonové spektrum, zlatý řez, konstanta jemné struktury).

\subsection{Shrnutí}

\begin{tcolorbox}[colback=blue!5!white, colframe=blue!75!black, title=\textbf{Klíčové výsledky}]

\textbf{Odvození Higgsovy VEV z QCT:}

\begin{enumerate}
\item \textbf{Základní vztah:}
\[
v \approx \Lambda_{\rm micro} \times \varphi^{12} = 236.02~\text{GeV} \quad (\text{chyba: } 4.14\%)
\]

\item \textbf{Elektromagnetická korekce:}
\[
v \approx \Lambda_{\rm micro} \times \varphi^{12 \times (1 + 1/\alpha_{\rm EM}^{-1})} = 246.18~\text{GeV} \quad (\text{chyba: } 0.015\%)
\]

\item \textbf{Fibonacciho dekompozice:}
\[
v \approx \Lambda_{\rm micro} \times (144\varphi + 89) \quad (F_{12} = 144,\, F_{11} = 89)
\]

\item \textbf{Alternativa (druhá odmocnina):}
\[
\sqrt{v} \approx \Lambda_{\rm micro} \times F_8 = 15.39~\text{GeV} \quad (F_8 = 21,\, \text{chyba: } 1.9\%)
\]

\item \textbf{Fyzikální interpretace:}
\begin{itemize}
\item Číslo 12 se vztahuje ke struktuře SM (3 generace, 4 dimenze, 12 kalibračních bosonů)
\item Elektroslaká škála emerguje prostřednictvím 12-krokové Fibonacciho hierarchie
\item Zlatý řez se objevuje jako optimalizační konstanta
\item Konstanta jemné struktury poskytuje EM korekci
\end{itemize}
\end{enumerate}

\end{tcolorbox}

\textbf{Důsledky:}

\begin{itemize}
\item Higgsova VEV \textbf{není libovolným parametrem}, ale emerguje z mikroskopické škály QCT prostřednictvím fundamentálních matematických konstant.

\item To spojuje QCT s \textbf{narušením elektroslaké symetrie} prostřednictvím geometrické hierarchie řízené zlatým řezem.

\item Výskyt $\varphi$ jak u $\Sigma$ baryonů ($\Lambda_{\rm micro}/m_\Sigma \approx 1/\varphi$, příloha~\ref{app:golden_ratio}) tak u Higgsovy VEV ($v/\Lambda_{\rm micro} \approx \varphi^{12}$) naznačuje \textbf{univerzální princip} řídící interakce neutrinového kondenzátu napříč energetickými škálami.

\item Pokud bude potvrzen lattice QCD a kosmologickými pozorováními, toto by představovalo \textbf{první úspěšné postdikční vysvětlení} Higgsovy VEV z mikroskopické teorie, s potenciálem stát se predikčním prostřednictvím testů kosmologické evoluce $v(z)$.
\end{itemize}

\subsection{Otevřené otázky}

\begin{enumerate}
\item Může teorie grup identifikovat pětiúhelníkovou podgrupu SU(3), která přirozeně produkuje $\varphi$ a $\varphi^{12}$?

\item Proč přesně 12 kroků? Existuje rekurzivní struktura v hierarchii neutrinového kondenzátu?

\item Lze vztah $\sqrt{v} \approx \Lambda_{\rm micro} \times F_8$ sladit s $v \approx \Lambda_{\rm micro} \times \varphi^{12}$?

\item Vzniká EM korekce $1/\alpha_{\rm EM}^{-1}$ z 1-smyčkové výměny fotonů, nebo z hlubšího principu?

\item Jak se $v(z)$ vyvíjí kosmologicky? Mohou data BBN a CMB testovat predikovanou trajektorii $v(z)$?

\item Existují jiné fundamentální konstanty (hmotnosti kvarků, směšovací úhly), které následují vzory zlatého řezu?
\end{enumerate}

\textbf{Doporučení:} Tento vzor zasluhuje věnované lattice QCD simulace, grupově-teoretickou analýzu a kosmologická omezení k validaci nebo vyvrácení hypotézy.
