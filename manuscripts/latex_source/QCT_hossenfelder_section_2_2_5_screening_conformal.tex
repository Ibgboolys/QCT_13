% NEW SECTION 2.2.5: Geometric Origin of Screening via Conformal Rescaling
% Location: Insert after Eq. 486 (screening factor derivation) in preprint.tex
% Priority: 1 (MUST HAVE)
% Length: ~1.5 pages
% Connection: Hossenfelder & Zingg (2020), Sec. 3

\subsubsection{Geometric Origin of Screening: Connection to Analogue Gravity}
\label{sec:screening_conformal}

The screening factor $f_{\rm screen} = m_\nu/m_p$ has been derived (Eq.~\ref{eq:screening_mass_ratio}) from both fundamental mass ratios and geometric considerations. Here we establish a deeper connection: \textbf{QCT screening is equivalent to conformal rescaling in analogue gravity}.

\paragraph{Conformal Rescaling Framework.}

Following Hossenfelder \& Zingg~\cite{Hossenfelder2020}, consider two metrics related by a conformal transformation:
\begin{equation}
\tilde{g}_{\mu\nu}(r) = \Omega^2(r) \cdot g_{\mu\nu}(r),
\label{eq:conformal_rescaling}
\end{equation}
where $\Omega(r)$ is the \emph{conformal factor}. In analogue gravity, this rescaling changes how perturbations of the condensate perceive the effective spacetime. Crucially, the effective mass of perturbations transforms as:
\begin{equation}
\tilde{m}^2_{\text{eff}} = \Omega^2 m^2_{\text{eff}} + \Omega^{(2-n)/2} \tilde{\Box} \Omega^{(n-2)/2},
\end{equation}
where $n=3$ is the number of spatial dimensions and $\tilde{\Box}$ is the d'Alembertian in the rescaled metric.

\paragraph{QCT Conformal Factor.}

We define the QCT conformal factor as:
\begin{equation}
\boxed{\Omega_{\text{QCT}}(r) = \sqrt{f_{\text{screen}} \cdot K(r)} = \sqrt{\frac{m_\nu}{m_p}} \cdot \sqrt{1 + \alpha\frac{\Phi(r)}{c^2}}}
\label{eq:QCT_conformal_factor}
\end{equation}
where $K(r) = 1 + \alpha \Phi(r)/c^2$ quantifies the local neutrino density enhancement (Eq.~\ref{eq:n_nu_local}) and $\alpha \approx -9 \times 10^{11}$ is the neutrino-gravitational coupling.

\paragraph{Equivalence to Yukawa Screening.}

The effective gravitational constant under conformal rescaling is:
\begin{equation}
G_{\text{eff}}(r) = \Omega_{\text{QCT}}^{-2}(r) \cdot G_N = \frac{G_N}{f_{\text{screen}} \cdot K(r)}.
\end{equation}

For the sub-millimeter regime where Yukawa screening dominates (Eq.~\ref{eq:G_eff_full}), this reproduces:
\begin{equation}
G_{\text{eff}}(r) \approx G_N \cdot \exp\left(-\frac{r}{\lambda_{\text{screen}}(r)}\right), \quad \text{where} \quad \lambda_{\text{screen}}(r) = \frac{R_{\text{proj}}(r)}{\ln(1/f_{\text{screen}})}.
\end{equation}

The connection is:
\begin{equation}
\lambda_{\text{screen}}(r) = \frac{R_{\text{proj}}^{(0)}}{\ln(1/f_{\text{screen}})} \cdot \Omega_{\text{QCT}}^{-1}(r) \cdot \sqrt{K(r)}.
\end{equation}

\paragraph{Physical Interpretation.}

\begin{itemize}
\item \textbf{Classical Analogue Gravity (Hossenfelder):} The conformal factor $\Omega(r)$ is introduced to satisfy fluid equations of motion (continuity + Euler). It represents a \emph{classical} reparametrization of perturbations.

\item \textbf{Quantum Analogue Gravity (QCT):} The conformal factor $\Omega_{\text{QCT}}(r)$ arises \emph{dynamically} from the environment-dependent coherence length of the neutrino condensate:
\begin{equation}
\xi(r) = \frac{\xi_0}{\sqrt{K(r)}} \quad \Rightarrow \quad R_{\text{proj}}(r) = R_{\text{proj}}^{(0)} \cdot \frac{1}{\sqrt{K(r)}} = R_{\text{proj}}^{(0)} \cdot \Omega_{\text{QCT}}^{-1}(r) / \sqrt{f_{\text{screen}}}.
\end{equation}

\item \textbf{Geometric Principle:} Screening is not phenomenological, but follows from the requirement that the neutrino condensate (described by the Gross-Pitaevskii equation) must simultaneously generate the metric \emph{and} satisfy its own equations of motion.
\end{itemize}

\paragraph{Environment-Dependent Screening Length.}

Combining Eq.~\ref{eq:QCT_conformal_factor} with the projection radius scaling (Eq.~\ref{eq:R_proj_environment}):
\begin{equation}
\lambda_{\text{screen}}(\mathbf{r}) = \lambda_{\text{screen}}^{(0)} \cdot \frac{1}{\sqrt{K(\mathbf{r})}},
\label{eq:lambda_screen_environment}
\end{equation}
where $\lambda_{\text{screen}}^{(0)} \approx 1.0$ mm is the cosmic baseline.

\textbf{Numerical verification:}
\begin{align}
\text{Earth:} \quad & K_\oplus = 625 \quad \Rightarrow \quad \lambda_{\text{screen}}^\oplus = \frac{1.0 \text{ mm}}{\sqrt{625}} = 40\,\mu\text{m} \quad \checkmark \\
\text{ISS (400 km):} \quad & K_{\text{ISS}} = 590 \quad \Rightarrow \quad \lambda_{\text{screen}}^{\text{ISS}} = \frac{1.0 \text{ mm}}{\sqrt{590}} = 41\,\mu\text{m} \quad \checkmark
\end{align}

\paragraph{Testable Prediction.}

The ratio of screening lengths between different gravitational environments is:
\begin{equation}
\frac{\lambda_{\text{screen}}^{\text{ISS}}}{\lambda_{\text{screen}}^{\oplus}} = \sqrt{\frac{K_\oplus}{K_{\text{ISS}}}} = \sqrt{\frac{625}{590}} \approx 1.029.
\end{equation}

\textbf{Experimental test:} Sub-millimeter gravity experiments on the ISS should measure $\lambda_{\text{screen}}^{\text{ISS}} \approx 41\,\mu\text{m}$, a $2.5\%$ increase compared to Earth-based measurements ($40\,\mu\text{m}$). This provides a direct test of the environment-dependent conformal factor.

\paragraph{Connection to Hossenfelder Framework.}

Hossenfelder \& Zingg~\cite{Hossenfelder2020} showed that introducing a conformal factor extends the class of metrics that can be realized as analogue gravity models. For a static black hole spacetime, they derive (their Eq.~33):
\begin{equation}
\Omega_{\text{Hossenfelder}}(r) = \frac{1}{r}\left[1-\gamma(r)\right]^{1/(n-1)}, \quad \gamma(r) = 1 - \frac{2GM}{r}.
\end{equation}

In QCT, the equivalent conformal factor for a gravitational source is:
\begin{equation}
\Omega_{\text{QCT}}(r) \sim \frac{1}{\sqrt{K(r)}} \sim \left[1 + \frac{GM}{r c^2}\right]^{-1/2} \quad \text{(weak field)}.
\end{equation}

Both frameworks use conformal rescaling to satisfy fluid equations, but QCT derives $\Omega(r)$ from \emph{quantum coherence} rather than classical parametrization.

\paragraph{Implications for Sub-Millimeter Gravity.}

The conformal interpretation clarifies the physical origin of sub-millimeter screening:

\begin{enumerate}
\item \textbf{Cosmic baseline ($\Phi \approx 0$):} $K \approx 1$, $\lambda_{\text{screen}} \approx 1$ mm (maximal coherence).
\item \textbf{Earth environment ($\Phi_\oplus \approx -6.25 \times 10^7$ m$^2$/s$^2$):} $K_\oplus \approx 625$, $\lambda_{\text{screen}}^\oplus \approx 40\,\mu$m (reduced coherence).
\item \textbf{Near compact objects (neutron stars, black holes):} $K \gg 10^{20}$, $\lambda_{\text{screen}} \ll 1$ nm (extreme decoherence).
\end{enumerate}

This resolves the apparent paradox: screening becomes \emph{stronger} (shorter $\lambda_{\text{screen}}$) in denser gravitational environments, as expected for a condensate responding to external potentials.

\paragraph{Summary.}

The screening factor $f_{\text{screen}} = m_\nu/m_p$ has a dual interpretation:
\begin{itemize}
\item \textbf{Microscopic:} Mass ratio quantifying neutrino-baryon coupling strength.
\item \textbf{Geometric:} Conformal rescaling factor determining how perturbations perceive spacetime curvature.
\end{itemize}

This connection to analogue gravity theory~\cite{Hossenfelder2020, Barcelo2005} transforms QCT screening from a phenomenological fit to a \textbf{geometric principle}, significantly strengthening its theoretical foundation.
