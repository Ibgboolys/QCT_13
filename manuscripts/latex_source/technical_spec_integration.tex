% Technical Specification: Complete Microscopic Derivation
% Integration into QCT Preprint
% Date: 2025-10-15

\section{Technical Specification of Microscopic Formalism}
\label{sec:technical_spec}

This section provides the complete set of equations for the microscopic derivation of QCT, which was developed in parallel with the EFT framework of the preprint. We show explicit parameter mappings and identify open questions requiring further research.

\subsection{Equation box for preprint}

\subsubsection{Box 1: Fundamental field}

\begin{tcolorbox}[colback=blue!5!white,colframe=blue!75!black,title=Neutrino Condensate Field]
\begin{equation}\label{eq:psi_def}
\boxed{\Psi_{\nu\nu}(\mathbf{x},t) = \sqrt{\rho_{\rm ent}(\mathbf{x},t)} \cdot e^{i\theta(\mathbf{x},t)}}
\end{equation}

\vspace{-0.2cm}
\begin{equation}\label{eq:schrodinger_full}
\boxed{i\hbar\frac{\partial\Psi_{\nu\nu}}{\partial t} = \left[-\frac{\hbar^2}{2m_\nu}\nabla^2 + \frac{\lambda}{4!}|\Psi_{\nu\nu}|^2 + V_{\rm ext}\right]\Psi_{\nu\nu} - i\frac{\Gamma_{\rm dec}}{2}\Psi_{\nu\nu}}
\end{equation}

where $V_{\rm ext} = \kappa_{\rm grav}\rho_m + \kappa_{\rm EM}|\mathbf{E}|^2$.
\end{tcolorbox}

\subsubsection{Box 2: Emergent Gravity}

\begin{tcolorbox}[colback=green!5!white,colframe=green!75!black,title=Metric and Gravitational Constant]
\begin{equation}\label{eq:metric_weak}
\boxed{g_{00} = -\left(1+\frac{2\Phi}{c^2}\right),\quad \Phi(\mathbf{x}) = -G_{\rm eff}\int d^3x'\,\frac{\rho_m(\mathbf{x}')}{|\mathbf{x}-\mathbf{x}'|}}
\end{equation}

\vspace{-0.2cm}
\begin{equation}\label{eq:G_eff_micro}
\boxed{G_{\rm eff} = \frac{c_\rho}{M_{\rm Pl}^2\Lambda_{\rm QCT}^2}\cdot\frac{n_\nu\Lambda_{\rm QCT}^2 V_{\rm proj}}{m_\nu R_{\rm proj}}}
\end{equation}

where $V_{\rm proj}=F_{\rm proj}/n_\nu$, $R_{\rm proj}=(3V_{\rm proj}/4\pi)^{1/3}$.
\end{tcolorbox}

\subsubsection{Box 3: Emergent Electromagnetism}

\begin{tcolorbox}[colback=red!5!white,colframe=red!75!black,title=Gauge fields and the Maxwell Lagrangian]
\begin{equation}\label{eq:gauge_phase}
\boxed{A_\mu = \frac{\hbar}{e_{\rm eff}}\partial_\mu\theta,\quad F_{\mu\nu}=\partial_\mu A_\nu-\partial_\nu A_\mu}
\end{equation}

\vspace{-0.2cm}
\begin{equation}\label{eq:e_eff_renorm}
\boxed{e_{\rm eff}^2 = e^2\cdot\sqrt{\frac{n_\nu\hbar^2}{\mu_0 c}},\quad \partial_\nu F^{\nu\mu}=\mu_0 J^\mu}
\end{equation}

\emph{Note:} The renormalization factor $\sim 10^{17}$ corresponds to the number of coherent neutrinos in $V_{\rm proj}$ — collective amplification!
\end{tcolorbox}

\subsubsection{Box 4: Numerical Parameters}

\begin{tcolorbox}[colback=yellow!5!white,colframe=orange!75!black,title=Input Values]
\begin{align}
&\Lambda_{\rm QCT}=107\,{\rm TeV},\quad m_\nu=0.1\,{\rm eV},\quad n_\nu=336\,{\rm cm}^{-3}, \label{eq:params1}\\
&\lambda\approx 6\times 10^{-2},\quad \frac{c_\rho}{\Lambda_{\rm QCT}^2}=3\times 10^{-12},\quad F_{\rm proj}=2.43\times 10^4. \label{eq:params2}
\end{align}
\end{tcolorbox}

\subsection{Mapping of Microscopic Parameters to EFT}

\begin{table}[h]
\centering
\caption{Complete Parameter Mapping.}
\begin{tabular}{lllll}
\toprule
\textbf{Microscopic} & \textbf{Value} & \textbf{EFT (preprint)} & \textbf{Value (EFT)} & \textbf{Relation} \\
\midrule
$F_{\rm proj}$ & $2.43\times 10^4$ & $\mathcal N_{\rm eff}$ & $\sim 10^4$ & identification \\
$E_{\rm pair}$ & $\Lambda_{\rm QCT}^2/m_\nu$ & — & — & gain \\
$K$ (stiffness) & $\sim 10^8$ Pa & — & — & $c^2=K/\rho$ \\
$\alpha$ (grav.) & $\sim 1$ & $\kappa_{\rm grav}/c_\rho$ & $\sim 1$ & coupling \\
$\lambda$ & $6\times 10^{-2}$ & $\lambda$ & $\sim 10^{-2}$ & fits \\
\bottomrule
\end{tabular}
\end{table}

\paragraph{Derivation of $\lambda$ from the coupling.}
The self-interaction constant is determined by the coupling energy:
\begin{equation}
\lambda = \frac{E_{\rm pair}^2}{\rho_{\rm ent}},
\end{equation}
where $E_{\rm pair}\sim\Lambda_{\rm QCT}^2/m_\nu$ and $\rho_{\rm ent}=(\lambda/24)n_\nu^2 m_\nu^2$. Solving for $\lambda$ (self-consistent) we get $\lambda\sim 10^{-2}$ — in agreement with the preprint.

\paragraph{Cutoff scale.}
Relationship between microscopic scale ($E_{pair}$) and EFT cutoff:
\begin{equation}
\Lambda_{\rm QCT} = \sqrt{E_{\rm pair}\cdot m_\nu} \approx 107\,{\rm TeV}.
\end{equation}

\subsection{Entanglement density}

\paragraph{vacuum energy}
The condensate has self-energy:

\begin{equation}\label{eq:rho_correct}
\boxed{\rho_{\rm ent}^{(0)} = \frac{\lambda}{24}n_\nu^2\,(m_\nu c^2)^2 / \Lambda_{\rm QCT}^2 \sim 10^{-63}\,{\rm GeV}^4}
\end{equation}
This is consistent with cosmology.

\paragraph{Effective density for deriving $G$.}
For gravity calculations we use the \emph{effective} density including the binding energy:
\begin{equation}
\rho_{\rm eff} = n_\nu\cdot E_{\rm pair}\,({\rm pro}\;G_{\rm eff}),
\end{equation}
but with a geometric suppression of $\alpha\sim 10^{-10}$ from the overlaps of the projection volumes.

\subsection{Numerical calculations and consistency}

\paragraph{Projection geometry.}
\begin{align}
V_{\rm proj}&=\frac{F_{\rm proj}}{n_\nu}=\frac{2.43\times 10^4}{3.36\times 10^8\,{\rm m}^{-3}}=7.23\times 10^{-5}\,{\rm m}^3=72.3\,{\rm cm}^3,\\
R_{\rm proj}&=\left(\frac{3V_{\rm proj}}{4\pi}\right)^{1/3}\approx 2.58\,{\rm cm}.
\end{align}

\paragraph{Condensate stiffness.}
From $c^2=K/\rho_{\rm eff}$ (where $\rho_{\rm eff}$ is the effective for the photon):
\begin{equation}
K = \rho_{\rm eff}\,c^2.
\end{equation}
Numerically:
\begin{equation}
K\sim 9\times 10^7\,{\rm Pa}\,({\rm expected}),
\end{equation}
but the raw calculation gives $\sim 10^{19}$ Pa — suggesting an error in the conversion of $\rho_{\rm eff}$.

\paragraph{Gravity constant — fit.}
Back fitting from $G=6.67\times 10^{-11}$ we get the necessary geometric factor:
\begin{equation}
\alpha_{\rm fit}\approx 1.9\times 10^{-10}.
\end{equation}
This small factor must arise from the overlapping geometry of the projection volumes — \textbf{open question no. 1}.

\subsection{Testable predictions (summary)}

\begin{table}[h]
\centering
\caption{Testable predictions of the QCT microscopic formalism.}
\begin{tabular}{lll}
\toprule
\textbf{Observable} & \textbf{Predictions} & \textbf{Status} \\
\midrule
$\Delta G/G$ (neutron stars) & $\sim 10^{-2}$ & Testable by binary pulsars \\
$\dot G/G$ (cosmology) & $\sim 10^{-10}\,{\rm yr}^{-1}$ & At the limit of LLR limits \\
Shapiro delay dispersion & $\sim 1\,{\rm ns}$ & Testable FRBs \\
Atomic clocks (Yb$^+$) & $\Delta E/E\sim 10^{-18}$ & At the limit of current precision \\
Oklo $\Delta\alpha/\alpha$ & $<10^{-10}$ & Consistent ($\Lambda_{\rm QCT}$ constant) \\
\bottomrule
\end{tabular}
\end{table}

\subsection{Open questions and future research}

\paragraph{Critical questions.}

\begin{enumerate}[label=\textbf{Q\arabic*:},leftmargin=2cm]
\item \textbf{Geometric factor $\alpha\sim 10^{-10}$.} \\
Why is such a small coupling necessary? Is it an artifact of the units, or the real physics of overlaps? \\
\emph{Hypothesis:} The projection volumes overlap only in a small fraction of space.

\item \textbf{Units $\rho_{\rm eff}$.} \\
How to correctly convert $n_\nu\cdot E_{\rm pair}$ to kg/m$^3$? The crude calculation gives $\sim 10^2$ kg/m$^3$ (absurd). \\
\emph{Possible answer:} $E_{\rm pair}$ is not rest mass, but virtual energy — need for an effective theory.

\item \textbf{Renormalization of $e_{\rm eff}$.} \\
The factor $\sim 10^{17}$ in charge — is it physically meaningful? \\
\emph{Answer:} Collective amplification over $N\sim V_{\rm proj}\cdot n_\nu\approx 2.4\times 10^4$ pairs. Maybe $e_{\rm eff}=e/\sqrt{N}$.

\item \textbf{Screening at extreme densities.} \\
How does the saturation of $\lambda|\Psi|^2$ work at $n_\nu\to\infty$ (neutron stars)? \\
\emph{Required:} Nonlinear analysis of GP equations in the high-density limit.
\end{enumerate}