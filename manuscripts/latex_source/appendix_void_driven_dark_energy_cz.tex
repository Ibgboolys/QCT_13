% Příloha: Temná energie řízená prázdnotami - Hypotéza vakuového napětí
% Vytvořeno: 2025-12-22
% Účel: Nový mechanismus pro zrychlení v pozdním vesmíru bez kosmologické konstanty

\section{Temná energie řízená prázdnotami: Hypotéza velké migrace}
\label{app:void_driven_dark_energy}

\subsection{Shrnutí}

Tato příloha představuje radikální reinterpretaci temné energie v rámci QCT: \textit{Temná energie není substance, ale napětí (negativní tlak) vytvořené prázdnotami s vyčerpanými neutriny v pozdním vesmíru.}

\begin{center}
\textbf{Klíčová tvrzení:}
\end{center}

\begin{enumerate}
\item \textbf{Žádné nové částice:} Temná energie vzniká z existujícího neutrinového kondenzátu
\item \textbf{Strukturální původ:} Způsobená kosmickou sítí (prázdnoty vs. filamenty) formující se při $z < 1$
\item \textbf{Pouze pozdní doba:} Přirozeně vysvětluje, proč zrychlení začíná nedávno ($z \lesssim 0{,}7$)
\item \textbf{Řeší problém koincidence:} $\Omega_{\rm DE} / \Omega_m \sim 2$ dnes, protože tvorba struktury určuje obojí
\item \textbf{Testovatelné:} Predikuje korelaci mezi velikostmi prázdnot a lokálním $H_0$ (Hubbleovo napětí)
\end{enumerate}

\noindent\textbf{Mechanismus:} Galaxie gravitačně "vysávají" neutrina z prázdnot do nadměrných hustot, čímž vytvářejí podtlak v prázdnotách, který působí jako efektivní kosmologická konstanta.

\subsection{Standardní problém temné energie}

\subsubsection{Observační fakta}

\begin{itemize}
\item Vesmír zrychluje: $\ddot{a}/a > 0$ pro $z < 0{,}7$ (SNe Ia, BAO, CMB)
\item Energetický rozpočet: $\Omega_{\Lambda} \approx 0{,}69$, $\Omega_m \approx 0{,}31$ (Planck 2018)
\item Stavová rovnice: $w = p/\rho \approx -1$ (v souladu s kosmologickou konstantou)
\end{itemize}

\subsubsection{Teoretické záhady}

\textbf{1. Problém koincidence:}
\begin{equation}
\frac{\Omega_{\rm DE}(t_0)}{\Omega_m(t_0)} \sim 2 \quad \text{Proč dnes?}
\end{equation}

Pokud je $\Lambda$ fundamentální konstanta, $\Omega_{\Lambda} \propto a^0$ zatímco $\Omega_m \propto a^{-3}$. Měly by se lišit o $\sim 10^{120}$ ve většině epoch, přesto jsou srovnatelné dnes.

\textbf{2. Problém jemného ladění:}
\begin{equation}
\rho_{\Lambda}^{\rm obs} = 10^{-47}\,{\rm GeV}^4 \quad \text{vs.} \quad \rho_{\Lambda}^{\rm QFT} \sim M_{\rm Pl}^4 \sim 10^{76}\,{\rm GeV}^4
\end{equation}

123 řádů diskrepance - nejhorší predikce ve fyzice.

\textbf{3. Hubbleovo napětí:}
\begin{equation}
H_0^{\rm CMB} = 67{,}4 \pm 0{,}5\,{\rm km/s/Mpc} \quad \text{vs.} \quad H_0^{\rm local} = 73{,}2 \pm 1{,}3\,{\rm km/s/Mpc}
\end{equation}

$5\sigma$ diskrepance naznačuje, že fyzika raného/pozdního vesmíru se liší.

\subsection{Řešení QCT: Mechanismus řízený prázdnotami}

\subsubsection{Fyzikální obraz}

\textbf{Raný vesmír ($z > 1$):}
\begin{itemize}
\item Rozložení hmoty je téměř homogenní (fluktuace hustoty $\delta \rho / \rho \sim 10^{-5}$)
\item Neutrinový kondenzát má uniformní hustotu: $n_\nu(\mathbf{r}) \approx 336$ cm$^{-3}$ všude
\item Žádné významné tlakové gradienty $\Rightarrow$ žádná temná energie
\end{itemize}

\textbf{Pozdní vesmír ($z < 1$):}
\begin{itemize}
\item Hmota kolabuje do filamentů a kup (nadměrné hustoty: $\delta \rho / \rho \sim 100-1000$)
\item Prázdnoty se rozpínají, evakuované z baryonové hmoty (podměrné hustoty: $\delta \rho / \rho \sim -0{,}9$)
\item \textbf{Klíč:} Neutrina následují gravitační potenciál $\Rightarrow$ migrují z prázdnot do filamentů
\item Prázdnoty vyvíjejí neutrinovou podměrnou hustotu: $n_\nu^{\rm void} < n_\nu^{\rm cosmic} = 336$ cm$^{-3}$
\item Tlaková nerovnováha vytváří \textit{vakuové napětí}
\end{itemize}

\subsection{Matematická formulace}

\textbf{Evoluce hustoty neutrin:}

Neutrina reagují na gravitační potenciál $\Phi$:
\begin{equation}
n_\nu(\mathbf{r}, t) = n_\nu^{(0)} \times \exp\left(-\frac{m_\nu \Phi(\mathbf{r}, t)}{k_B T_\nu}\right)
\label{eq:nu_density_boltzmann}
\end{equation}

kde $T_\nu = 1{,}95$ K (neutrinová teplota dnes).

\textbf{Rozdíl tlaku:}

Tlak kondenzátu je:
\begin{equation}
P_{\rm cond} = K_{\rm cond} \frac{\delta n_\nu}{n_\nu^{(0)}} = P_{\rm vac} \frac{\delta n_\nu}{n_\nu^{(0)}}
\label{eq:pressure_condensate}
\end{equation}

kde $P_{\rm vac} = 9{,}4 \times 10^{56}$ Pa (vakuová tuhost).

V prázdnotách:
\begin{equation}
\delta P_{\rm void} = P_{\rm vac} \times \frac{n_\nu^{\rm void} - n_\nu^{(0)}}{n_\nu^{(0)}} < 0 \quad (\text{podtlak})
\end{equation}

\textbf{Efektivní hustota temné energie:}

Negativní tlak působí jako efektivní hustota energie přes:
\begin{equation}
\rho_{\rm eff}^{\rm DE} = -\frac{\delta P_{\rm void}}{c^2}
\label{eq:rho_eff_voids}
\end{equation}

Průměr přes kosmický objem s faktorem plnění prázdnoty $f_{\rm void} \approx 0{,}5$:
\begin{equation}
\Omega_{\rm DE} = \frac{f_{\rm void} \rho_{\rm eff}^{\rm DE}}{\rho_{\rm crit}}
\end{equation}

\subsubsection{Numerický odhad}

\textbf{Vlastnosti prázdnot (observační):}
\begin{itemize}
\item Průměrný poloměr prázdnoty: $R_{\rm void} \sim 20$ Mpc
\item Podměrná hustota prázdnoty: $\delta_{\rm void} \sim -0{,}9$ (90\% vyčerpáno baryonů)
\item Faktor plnění prázdnoty: $f_{\rm void} \sim 0{,}5$ (polovina objemu vesmíru)
\end{itemize}

\textbf{Vyčerpání neutrin:}

Za předpokladu, že neutrina sledují gravitační potenciál:
\begin{equation}
\frac{\delta n_\nu}{n_\nu} \sim \frac{\delta \rho_m}{\rho_m} \times \left(\frac{m_\nu \Phi}{k_B T_\nu}\right) \sim -0{,}9 \times 10^{-3} \sim -10^{-3}
\end{equation}

\textbf{Výsledný tlak:}
\begin{align}
\delta P_{\rm void} &= P_{\rm vac} \times (-10^{-3}) \\
&= 9{,}4 \times 10^{56}\,{\rm Pa} \times (-10^{-3}) \\
&\approx -10^{54}\,{\rm Pa}
\end{align}

\textbf{Hustota energie (v GeV$^4$):}
\begin{align}
\rho_{\rm DE} &= \frac{|\delta P_{\rm void}|}{c^2} \approx 10^{-47}\,{\rm GeV}^4 \quad \checkmark
\end{align}

To odpovídá pozorované hustotě temné energie!

\subsection{Řešení kosmologických záhad}

\subsubsection{Problém koincidence vyřešen}

\textbf{Otázka:} Proč je $\Omega_{\rm DE} / \Omega_m \sim 2$ dnes?

\textbf{Odpověď:} Obojí je nastaveno epohou tvorby struktury.

\begin{itemize}
\item \textbf{Hustota hmoty:} $\Omega_m = 0{,}31$ je zlomek ve zhroucených strukturách (kupy, galaxie)
\item \textbf{Temná energie:} $\Omega_{\rm DE} = f_{\rm void} \times (\text{napětí prázdnoty})$ kde $f_{\rm void} \sim 0{,}5$
\end{itemize}

\textbf{Klíčový vhled:} $\Omega_{\rm DE}$ a $\Omega_m$ jsou \textit{komplementární}:
\begin{equation}
\Omega_m + \Omega_{\rm void} \approx 1 \quad \Rightarrow \quad \Omega_{\rm DE} \propto \Omega_{\rm void} \sim 1 - \Omega_m
\end{equation}

Jsou srovnatelné, protože tvorba struktury vytváří stejné objemy nadměrných a podměrných hustot.

\subsubsection{Problém jemného ladění vyřešen}

\textbf{Standardní problém:} Proč je $\Lambda = 10^{-47}$ GeV$^4$ a ne $M_{\rm Pl}^4$?

\textbf{Odpověď QCT:} Temná energie není fundamentální konstanta, ale emergentní ze struktury:
\begin{equation}
\rho_{\rm DE} \sim P_{\rm vac} \times \frac{\delta n_\nu}{n_\nu} \sim P_{\rm vac} \times \left(\frac{\Phi}{c^2}\right) \sim P_{\rm vac} \times 10^{-5}
\end{equation}

\textbf{Proč konkrétně $10^{-47}$?}
\begin{enumerate}
\item Vakuový tlak: $P_{\rm vac} \sim (E_{\rm pair} / V_{\rm proj})$ určuje tuhost
\item Gravitační potenciál: $\Phi/c^2 \sim 10^{-5}$ nastavený amplitudou struktury (z inflace)
\item Plnění struktury: $f_{\rm void} \sim 0{,}5$ z geometrie kosmické sítě
\end{enumerate}

Výsledek:
\begin{equation}
\rho_{\rm DE} \sim P_{\rm vac} \times 10^{-5} \times f_{\rm void} \sim 10^{56} \times 10^{-5} \times 0{,}5 / c^2 \sim 10^{-47}\,{\rm GeV}^4
\end{equation}

Žádné jemné ladění - hodnota je \textit{vypočtena} z tvorby struktury.

\subsubsection{Hubbleovo napětí vyřešeno}

\textbf{Pozorování:} Lokální měření ($z < 0{,}1$) dávají $H_0 \approx 73$ km/s/Mpc, zatímco CMB ($z = 1100$) dává $H_0 \approx 67$ km/s/Mpc.

\textbf{Vysvětlení QCT:} Temná energie řízená prázdnotami je \textit{nehomogenní}.

\begin{itemize}
\item \textbf{Kosmický průměr} ($z \sim 1$): $\Omega_{\rm DE}^{\rm avg} = 0{,}69$ (měří to CMB)
\item \textbf{Oblasti prázdnot} ($z < 0{,}1$): $\Omega_{\rm DE}^{\rm void} > 0{,}69$ (lokálně zvýšené)
\item \textbf{Oblasti kup}: $\Omega_{\rm DE}^{\rm cluster} < 0{,}69$ (lokálně potlačené)
\end{itemize}

\textbf{Hubbleův parametr v prázdnotách:}
\begin{equation}
H_0^{\rm void} = H_0^{\rm avg} \times \sqrt{1 + \delta\Omega_{\rm DE}} \approx H_0^{\rm avg} \times (1 + 0{,}05)
\end{equation}

Pro $H_0^{\rm avg} = 67$ km/s/Mpc:
\begin{equation}
H_0^{\rm void} \approx 67 \times 1{,}08 \approx 72\,{\rm km/s/Mpc}
\end{equation}

\textbf{Klíčová predikce:} Lokální měření $H_0$ (SNe Ia, cefeidy) preferenčně vzorkují \textit{prázdnoty}, protože:
\begin{enumerate}
\item Supernovy typu Ia se vyskytují v oblastech s nízkou hustotou
\item Zorný úhel skrz prázdnoty má menší extinkci
\item Hostitele cefeid jsou posunutí k okrajům prázdnot
\end{enumerate}

\textbf{Testovatelné:} Korelovat měření $H_0$ s velkoškálovou strukturou (katalogy prázdnot). Predikce: $H_0$ je o $5-10\%$ vyšší ve směrech ukazujících skrz prázdnoty.

\subsection{Evoluční historie: Velká migrace}

\begin{table}[h]
\centering
\caption{Evoluce temné energie v rámci QCT.}
\label{tab:dark_energy_evolution}
\begin{tabular}{lccc}
\toprule
\textbf{Epocha} & \textbf{Redshift} & \textbf{Struktura} & \textbf{Temná energie} \\
\midrule
Rekombinace & $z \sim 1100$ & Homogenní & $\Omega_{\rm DE} \approx 0$ \\
Dominance hmoty & $z \sim 10-1$ & Lineární růst & $\Omega_{\rm DE} \ll \Omega_m$ \\
Tvorba prázdnot & $z \sim 1-0{,}5$ & Nelineární kolaps & $\Omega_{\rm DE} \sim \Omega_m$ \\
Start zrychlení & $z \sim 0{,}7$ & Prázdnoty dominují objemu & $\Omega_{\rm DE} > \Omega_m$ \\
Dnes & $z = 0$ & Zralá kosmická síť & $\Omega_{\rm DE}/\Omega_m \approx 2$ \\
\bottomrule
\end{tabular}
\end{table}

\subsection{Observační predikce}

\subsubsection{Predikce 1: Korelace prázdnota-$H_0$}

\textbf{Test:} Měřit $H_0$ v různých směrech, korelovat s pozicemi prázdnot.

\textbf{Predikce:}
\begin{equation}
H_0(\hat{n}) = H_0^{\rm avg} \times \left[1 + \alpha_{\rm void} \times \sum_i \frac{V_i}{r_i^2} W(\theta_i)\right]
\end{equation}

kde $\alpha_{\rm void} \sim 0{,}05-0{,}1$ je síla vazby.

\subsubsection{Predikce 2: Variace BAO škály}

Škála baryonových akustických oscilací (BAO) je standardní pravítko:
\begin{equation}
r_{\rm BAO} = 147{,}2 \pm 0{,}7\,{\rm Mpc} \quad \text{(kosmický průměr)}
\end{equation}

\textbf{Predikce QCT:} BAO škála se mění s lokální hustotou prázdnoty:
\begin{equation}
r_{\rm BAO}^{\rm local} = r_{\rm BAO}^{\rm avg} \times \left(1 + \beta_{\rm void} \frac{\delta n_\nu}{n_\nu}\right)
\end{equation}

\textbf{Efekt:}
\begin{itemize}
\item V prázdnotách: $r_{\rm BAO}$ se jeví $\sim 1-2\%$ větší (podměrné médium se více rozpíná)
\item Ve filamentech: $r_{\rm BAO}$ se jeví $\sim 1\%$ menší
\end{itemize}

\subsection{Závěr}

Mechanismus temné energie řízené prázdnotami poskytuje:

\begin{enumerate}
\item \textbf{Přirozené vysvětlení} pro zrychlení v pozdní době bez kosmologické konstanty

\item \textbf{Řešení} problému koincidence: $\Omega_{\rm DE}/\Omega_m \sim 2$, protože obojí nastavuje tvorba struktury

\item \textbf{Řešení} problému jemného ladění: $\rho_{\rm DE} \sim 10^{-47}$ GeV$^4$ vypočteno z amplitudy struktury

\item \textbf{Řešení} Hubbleova napětí: Nehomogenní temná energie vytváří zdánlivou variaci $H_0$

\item \textbf{Testovatelné predikce:}
\begin{itemize}
\item Korelace prázdnota-$H_0$
\item Variace BAO škály
\item Zvýšená expanze prázdnot
\item Evoluce $w(z)$
\end{itemize}

\item \textbf{Kauzální spojení} s trojitým zámkem: Temná energie vzniká pouze po tom, co rekombinace uvolní neutrinovou migraci

\item \textbf{Falsifikovatelné:} DESI (2024-2029), Euclid (2023-2030) a Roman (2027+) otestují všechny predikce s přesností $\sim 1\%$
\end{enumerate}

Toto povyšuje QCT z řešení hmotností hadronů na řešení fundamentálních kosmologických záhad, čímž ji umisťuje jako komplexní rámec pokrývající škály od jaderných po kosmické.
