% ==============================================================================
% PŘÍLOHA Q: POZOROVACÍ OMEZENÍ Z KOSMOLOGICKÝCH DAT
% ==============================================================================
% Přímé srovnání predikcí QCT s daty Planck 2018 a DESI Y1
% Statistická analýza, χ² fitování, omezení parametrického prostoru
%
% Datum: 2025-12-11
% Stav: Datově řízená kvantitativní analýza
% Kód: simulations/cosmology/qct_vs_planck_data_comparison.py
%       simulations/cosmology/qct_vs_bao_data_comparison.py
% ==============================================================================

\section{Pozorovací omezení z kosmologických dat}
\label{app:observational_constraints}

Tato příloha představuje kvantitativní konfrontaci predikcí QCT s nejmodernějšími kosmologickými měřeními z Planck 2018~\cite{Planck2018} a DESI Year 1~\cite{DESI2024}. Na rozdíl od kvalitativních argumentů v sekcích~\ref{sec:cmb_phase_shift} a~\ref{sec:bao_consistency} provádíme rigorózní $\chi^2$ statistickou analýzu pomocí reálných pozorovacích dat.

\subsection{Motivace: Od kvalitativního ke kvantitativnímu}

Rámec QCT vytváří specifické predikce pro kosmologické pozorovatelné veličiny prostřednictvím dvou komplementárních mechanismů:

\begin{enumerate}
\item \textbf{Modifikovaná gravitace}: $G_{\rm eff} = (1 - \sigma^2) G_N \approx 0.9\,G_N$ na astrofyzikálních škálách (sekce~\ref{sec:astro_validation})
\item \textbf{Evoluce stavové rovnice}: Proměnné $w(z)$ z fázových přechodů neutrinového kondenzátu (příloha~\ref{app:dark_energy})
\end{enumerate}

Tyto mechanismy ovlivňují:
\begin{itemize}
\item \textbf{CMB}: Integrovaný Sachsův-Wolfeův (ISW) efekt při nízkém~$\ell$ ($\ell < 30$), historii expanze $H(z)$
\item \textbf{BAO}: Zvukový horizont $r_s$, úhlovou průměrovou vzdálenost $D_A(z)$, dilatační škálu $D_V(z)$
\end{itemize}

Pro posouzení životaschopnosti musíme porovnat kvantitativní predikce s přesnými měřeními.

\subsection{Fenomenologický model pro $w(z)$}

Podle přílohy~\ref{app:dark_energy} efektivní parametr stavové rovnice $w(z)$ vzniká z objemově průměrované dominance gradientů neutrinového kondenzátu:
\begin{equation}
w_{\rm eff}(z) = -\frac{1}{1 + X(z)^\alpha},
\label{eq:w_phenomenological}
\end{equation}
kde $X(z)$ kvantifikuje poměr gradientní energie k potenciální energii v poli kondenzátu $\Psi$.

\paragraph{Závislost na tvorbě struktur.}
Jak se kosmické struktury tvoří v pozdních časech, lokální gradienty $\nabla\Psi$ narůstají, zvyšující $X(z)$. Modelujeme to jako:
\begin{equation}
X(z) = X_0 \times \exp\left(-\frac{z}{z_{\rm structure}}\right),
\label{eq:X_of_z}
\end{equation}
kde:
\begin{itemize}
\item $X_0$: Dnešní ($z=0$) dominance gradientů (kosmický průměr)
\item $z_{\rm structure}$: Charakteristický červený posuv tvorby struktur
\item $\alpha$: Parametr ostrosti přechodu
\end{itemize}

\textbf{Fyzikální interpretace:}
\begin{itemize}
\item \textbf{Vysoké $z$ ($z \gg z_{\rm structure}$):} Vesmír homogenní $\Rightarrow$ $X \to 0$ $\Rightarrow$ $w \to -1$ (čistá temná energie)
\item \textbf{Nízké $z$ ($z \ll z_{\rm structure}$):} Struktury se tvoří $\Rightarrow$ $X \sim X_0$ $\Rightarrow$ $w > -1$ (odchylka od $\Lambda$)
\end{itemize}

\subsection{Omezení Planck 2018 CMB}

\subsubsection{Data a metodologie}

Porovnáváme predikce QCT s kosmologickými parametry Planck 2018~\cite{Planck2018}:
\begin{itemize}
\item \textbf{Stavová rovnice temné energie}: $w_0 = -1.03 \pm 0.03$ (68\% CL, TT,TE,EE+lowE+lensing+BAO)
\item \textbf{Hubbleův parametr}: $H_0 = 67.36 \pm 0.54$ km/s/Mpc
\item \textbf{Hustota hmoty}: $\Omega_m = 0.3153 \pm 0.0073$
\end{itemize}

Dodatečně používáme měření $H(z)$ z BOSS (odvozené z BAO)~\cite{BOSS2017}:
\begin{align}
z &= [0.38, 0.51, 0.61, 2.34] \\
H(z) &= [83.0, 90.4, 97.3, 222.0] \pm [2.5, 2.0, 2.1, 7.0] \, {\rm km/s/Mpc}
\end{align}

\subsubsection{Implementace QCT}

Evoluce hustoty temné energie v QCT následuje:
\begin{equation}
\rho_{\rm DE}(z) = \rho_{\rm DE}(0) \times \exp\left[3 \int_0^z \frac{1 + w(z')}{1 + z'} dz'\right],
\label{eq:rho_DE_evolution}
\end{equation}
což modifikuje Friedmannovu rovnici:
\begin{equation}
E^2(z) = \frac{H^2(z)}{H_0^2} = \Omega_{r,0}(1+z)^4 + \Omega_{m,0}(1+z)^3 + \Omega_{\Lambda,0} \frac{\rho_{\rm DE}(z)}{\rho_{\rm DE}(0)}.
\label{eq:E_QCT}
\end{equation}

\paragraph{Extrakce CPL parametrizace.}
Pro srovnání s omezeními Planck extrahujeme efektivní CPL parametry~\cite{ChevallierPolarski2001,Linder2003}:
\begin{equation}
w(a) = w_0 + w_a (1 - a) = w_0 + w_a \frac{z}{1+z},
\label{eq:CPL}
\end{equation}
fitováním rovnice~\eqref{eq:w_phenomenological} při nízkých červených posuvech ($z < 2$).

\subsubsection{Výsledky: Současná sada parametrů}

Používajíce explorační parametry z přílohy~\ref{app:dark_energy} ($X_0 = 10$, $z_{\rm structure} = 2$, $\alpha = 0.6$), zjišťujeme:

\begin{table}[h]
\centering
\caption{Efektivní parametry QCT vs omezení Planck 2018.}
\label{tab:qct_vs_planck_params}
\begin{tabular}{lccc}
\toprule
\textbf{Parametr} & \textbf{QCT (rovnice~\ref{eq:w_phenomenological})} & \textbf{Planck 2018} & \textbf{Napětí} \\
\midrule
$w_0$ & $-0.201$ & $-1.03 \pm 0.03$ & $27.6\,\sigma$ \\
$w_a$ & $-0.105$ & $-0.05 \pm 0.3$ & $0.2\,\sigma$ \\
$H_0$ [km/s/Mpc] & 67.36 (fixováno) & $67.36 \pm 0.54$ & — \\
$\Omega_m$ & 0.3153 (fixováno) & $0.3153 \pm 0.0073$ & — \\
\bottomrule
\end{tabular}
\end{table}

\textbf{Statistická analýza:}
\begin{itemize}
\item $\chi^2_{\rm QCT}(H(z)) = 555.1$ pro 4 datové body $\Rightarrow$ $\chi^2/{\rm dof} = 138.8$
\item $\chi^2_{\Lambda{\rm CDM}}(H(z)) = 6.0$ pro 4 datové body $\Rightarrow$ $\chi^2/{\rm dof} = 1.5$
\item $\Delta\chi^2 = 549.1 \gg 9$ $\Rightarrow$ \textbf{QCT silně nevýhodné při $>10\sigma$}
\end{itemize}

\paragraph{ISW amplituda.}
Příspěvek integrovaného Sachsova-Wolfeova efektu k CMB $C_\ell^{TT}$ při nízkém~$\ell$ je potlačen v QCT kvůli modifikované evoluci $\Phi$. Naše analýza (podsekce~\ref{subsec:isw_calculation}) dává:
\begin{equation}
\frac{C_\ell^{\rm ISW, QCT}}{C_\ell^{\rm ISW, \Lambda{\rm CDM}}} \approx 0.23 \pm 0.05,
\label{eq:ISW_ratio}
\end{equation}
ve srovnání s pozorovacím omezením~\cite{PlanckCollaboration2020}:
\begin{equation}
\frac{C_\ell^{\rm ISW, obs}}{C_\ell^{\rm ISW, \Lambda{\rm CDM}}} = 1.00 \pm 0.15.
\end{equation}

To představuje napětí $\sim 5\sigma$.

\subsubsection{Interpretace}

\textbf{Verdikt}: Současné fenomenologické parametry ($X_0 = 10$, $z_{\rm structure} = 2$, $\alpha = 0.6$) jsou \textbf{nekompatibilní s daty Planck 2018} při vysoké významnosti ($>10\sigma$).

To \textbf{neinvaliduje} fundamentální fyziku QCT (neutrinový kondenzát, fázové přechody), ale indikuje, že:
\begin{enumerate}
\item Tyto parametry \textbf{nebyly odvozeny} z kosmologických omezení, ale zvoleny jako počáteční odhady
\item Je vyžadována \textbf{optimalizace parametrů} pro shodu s pozorováními
\item Model je \textbf{falzifikovatelný}—síla pro vědeckou rigoróznost
\end{enumerate}

\subsection{Omezení DESI Year 1 BAO}

\subsubsection{Data a metodologie}

Data Dark Energy Spectroscopic Instrument (DESI) Year 1~\cite{DESI2024} poskytují nejpřesnější BAO měření k dnešnímu dni napříč šesti červenými posuvovými biny:
\begin{equation}
z = [0.295, 0.510, 0.706, 0.930, 1.317, 2.330]
\end{equation}

BAO pozorovatelná veličina je izotropní dilatační škála:
\begin{equation}
D_V(z) = \left[(1+z)^2 D_A^2(z) \frac{cz}{H(z)}\right]^{1/3},
\label{eq:DV_definition}
\end{equation}
kde $D_A(z) = D_C(z)/(1+z)$ je úhlová průměrová vzdálenost a $D_C(z)$ pohybující se vzdálenost:
\begin{equation}
D_C(z) = \frac{c}{H_0} \int_0^z \frac{dz'}{E(z')}.
\label{eq:DC_integral}
\end{equation}

DESI měří $D_V(z) / r_d$, kde $r_d$ je zvukový horizont při drag epoše.

\subsubsection{Implementace QCT}

V QCT jsou modifikovány jak $E(z)$ (rovnice~\ref{eq:E_QCT}) tak zvukový horizont $r_d$:

\paragraph{Modifikovaný zvukový horizont.}
Při drag epoše ($z_{\rm drag} \approx 1059$), pokud $G_{\rm eff} = 0.9\,G_N$ (sekce~\ref{sec:astro_validation}):
\begin{equation}
r_s^{\rm QCT} = \int_{z_{\rm drag}}^\infty \frac{c_s(z')}{H_{\rm QCT}(z')} dz' = \sqrt{\frac{G_N}{G_{\rm eff}}} \times r_s^{\Lambda{\rm CDM}} \approx 1.054 \, r_s^{\Lambda{\rm CDM}}.
\label{eq:rs_QCT}
\end{equation}

\paragraph{Modifikovaná měření vzdáleností.}
Úhlová průměrová vzdálenost $D_A(z)$ i Hubbleův parametr $H(z)$ jsou ovlivněny evolucí $w(z)$ prostřednictvím rovnice~\eqref{eq:E_QCT}.

\subsubsection{Výsledky: Současná sada parametrů}

Numerická integrace rovnic~\eqref{eq:DC_integral}--\eqref{eq:DV_definition} s $w(z)$ z rovnice~\eqref{eq:w_phenomenological} dává:

\begin{table}[h]
\centering
\caption{QCT vs $\Lambda$CDM: zlomkové odchylky v BAO pozorovatelných veličinách.}
\label{tab:qct_vs_lcdm_bao}
\small
\begin{tabular}{lcccc}
\toprule
\textbf{Červený posuv} & \textbf{$D_V^{\rm QCT}$} & \textbf{$D_V^{\Lambda{\rm CDM}}$} & \textbf{$\Delta D_V / D_V$} & \textbf{DESI $\sigma$} \\
 & [Mpc] & [Mpc] & [\%] & [typické] \\
\midrule
0.295 & 1062 & 1202 & $-11.6$ & 1.8\% \\
0.510 & 1562 & 1857 & $-15.9$ & 1.2\% \\
0.706 & 1958 & 2405 & $-18.6$ & 1.5\% \\
0.930 & 2422 & 3063 & $-20.9$ & 1.6\% \\
1.317 & 2975 & 3839 & $-22.5$ & 1.6\% \\
2.330 & 3365 & 4358 & $-22.8$ & 2.4\% \\
\bottomrule
\end{tabular}
\end{table}

\textbf{Statistická analýza:}
\begin{align}
\chi^2_{\rm QCT}(\text{DESI}) &= 1523.6 \quad (6 \, {\rm binů}) \label{eq:chi2_qct_desi} \\
\chi^2_{\Lambda{\rm CDM}}(\text{DESI}) &= 211.8 \quad (6 \, {\rm binů}) \label{eq:chi2_lcdm_desi} \\
\Delta\chi^2 &= 1311.8 \gg 9
\end{align}

Použitím Wilksovy věty odpovídá $\Delta\chi^2 > 9$ vyloučení $>3\sigma$.

\textbf{Reziduální graf} (dostupný ve výstupu kódu):
Všech šest DESI binů ukazuje systematické negativní reziduály $-10$ až $-20\sigma$, indikující, že QCT se současnými parametry predikuje vzdálenosti \textit{konzistentně kratší} než pozorované.

\subsubsection{Interpretace}

\textbf{Verdikt}: Současné parametry QCT jsou \textbf{vyloučeny při $>30\sigma$} daty DESI Y1.

\textbf{Fyzikální diagnóza}:
\begin{itemize}
\item Zlomkové odchylky $\Delta D_V / D_V \sim -15\%$ až $-25\%$
\item DESI přesnost: $\sim 1\%$--$2\%$
\item \textbf{QCT překračuje chybové úsečky faktorem 10--20}
\end{itemize}

\textbf{Implikace}: Fenomenologické parametry ($X_0 = 10$, $z_{\rm structure} = 2$, $\alpha = 0.6$) produkují \textit{příliš velké} odchylky od $\Lambda$CDM pro kompatibilitu s pozorováními.

\subsection{Exploraceč parametrického prostoru}

\subsubsection{Povolené rozsahy parametrů}

Pro dosažení kompatibility s Planck a DESI v rámci $2\sigma$ požadujeme:
\begin{align}
|w_0 + 1| &< 0.06 \quad (\text{Planck } 2\sigma) \label{eq:w0_constraint} \\
|\Delta D_V / D_V| &< 0.03 \quad (\text{DESI } 2\sigma \text{ při } z \sim 0.5) \label{eq:DV_constraint}
\end{align}

Použitím rovnice~\eqref{eq:w_phenomenological} s fixovaným $\alpha = 0.6$ se to překládá na:
\begin{align}
X_0 &< 0.05 \quad (\text{z rovnice~\ref{eq:w0_constraint}}) \\
z_{\rm structure} &> 20 \quad (\text{z rovnice~\ref{eq:DV_constraint}})
\end{align}

\textbf{Interpretace}:
\begin{itemize}
\item \textbf{Slabší dominance gradientů}: $X_0 \sim 0.01$--$0.05$ vs současné $X_0 = 10$ (redukce faktorem 200--1000)
\item \textbf{Pomalejší evoluce struktur}: $z_{\rm structure} \sim 20$--50 vs současné $z_{\rm structure} = 2$ (zvýšení faktorem 10--25)
\end{itemize}

\paragraph{Fyzikální zdůvodnění.}
Tato omezení implikují:
\begin{enumerate}
\item \textbf{Režim malých odchylek}: Efekty QCT na úrovni $\sim 0.1$--$1\%$, ne $\sim 10$--$20\%$
\item \textbf{Pozdní fenomén}: Efekt tvorby struktur se stává významným pouze při $z < 0.05$ (velmi nedávno)
\item \textbf{Testovatelné budoucími průzkumy}: Euclid, DESI 5-year, CMB-S4
\end{enumerate}

\subsubsection{Bayesovský výběr modelu}

Úplná Bayesovská analýza (mimo rozsah této přílohy) by vypočetla Bayesův faktor:
\begin{equation}
\mathcal{B}_{\rm QCT}^{\Lambda{\rm CDM}} = \frac{P({\rm data} | {\rm QCT})}{P({\rm data} | \Lambda{\rm CDM})},
\end{equation}
marginalizující přes parametrové priory $P(X_0, z_{\rm structure}, \alpha)$.

\textbf{Předběžný odhad}:
Se současnými parametry $\ln \mathcal{B} \approx -\Delta\chi^2 / 2 \approx -660$ (Planck) a $\approx -656$ (DESI), indikující \textbf{silný důkaz proti QCT}.

Avšak pokud optimalizované parametry dosáhnou $|\Delta\chi^2| < 4$, QCT by bylo \textbf{konkurenceschopné s $\Lambda$CDM}, nabízející fyzikální vysvětlení temné energie bez jemného doladění.

\subsection{Testovatelné predikce pro budoucí experimenty}

\subsubsection{CMB-S4 (2030s)}

\textbf{Cílová přesnost}: $\delta w_0 \sim 0.03$, ISW amplituda $\delta A_\infty \sim 0.001$

\textbf{Predikce QCT} (s optimalizovaným $X_0 < 0.05$):
\begin{itemize}
\item $w_0 \approx -0.99$ až $-1.00$ (v rámci $1\sigma$ $\Lambda$CDM)
\item ISW poměr: $0.95$--$1.00$ (sub-procentní odchylka)
\item \textbf{Rozlišující signatura}: Slabá škálová závislost v ISW křížové korelaci s LSS
\end{itemize}

\subsubsection{Euclid + DESI 5-Year (2025--2030)}

\textbf{Cílová přesnost}: $\Delta D_V / D_V \sim 0.1$--$0.3\%$ při $z < 2$

\textbf{Predikce QCT}:
\begin{itemize}
\item Konzistentní vzor odchylek napříč všemi $z$ biny
\item Závislost na červeném posuvu $\propto \exp(-z/z_{\rm structure})$
\item Na rozdíl od $N_{\rm eff}$ modelů: \textit{odlišné} signatury v CMB vs BAO
\end{itemize}

\subsubsection{Roman Space Telescope (2027)}

\textbf{Cílová přesnost}: $w_0$, $w_a$ z Type Ia SNe na $\sim 3\%$

\textbf{Predikce QCT}:
\begin{itemize}
\item $w(z)$ evoluce měřitelná pokud $z_{\rm structure} < 10$
\item Křížová kontrola se slabou čočkou $\Sigma(z)$ (rychlost růstu)
\end{itemize}

\subsection{Omezení a výhrady}

\subsubsection{Fenomenologická povaha}

Model rovnice~\eqref{eq:w_phenomenological} je \textbf{fenomenologický}, ne mikroskopicky odvozený. Parametry $(X_0, z_{\rm structure}, \alpha)$ kódují komplexní fyziku:
\begin{itemize}
\item Objemově průměrovaná gradientní energie $\langle |\nabla\Psi|^2 \rangle$
\item Nelineární tvorba struktur (kolaps hal, filamenty, prázdnoty)
\item Zpětná vazba baryonů na neutrinový kondenzát
\end{itemize}

\textbf{Odvození z prvních principů} by vyžadovalo:
\begin{enumerate}
\item Modifikovaný Boltzmannův kód (CAMB/CLASS + QCT)
\item N-body simulace s QCT gravitací
\item Efektivní teorii pole velkoškálové struktury (EFTofLSS) adaptovanou na QCT
\end{enumerate}

\subsubsection{Separace prostorových vs časových efektů}

Tato příloha zachází s $w(z)$ jako s \textit{pozaďovou} veličinou (průměrovanou přes prostor). Avšak QCT predikuje \textit{obojí}:
\begin{itemize}
\item \textbf{Prostorová variace}: $w({\bf r})$ z lokálních gradientů (galaxie, kupy)
\item \textbf{Časová evoluce}: $w(z)$ z tvorby struktur
\end{itemize}

Korektní zpracování vyžaduje perturbační teorii druhého řádu k vyhnutí se dvojitému počítání. Současná analýza předpokládá, že tyto efekty se čistě separují—zjednodušení vyžadující verifikaci.

\subsubsection{Degenerace modifikované gravitace}

Pozorovaná napětí by mohla být alternativně vysvětlena:
\begin{itemize}
\item Odlišnou evolucí $G_{\rm eff}(z)$ než předpokládaná konstantní $0.9\,G_N$
\item Škálově závislou $G_{\rm eff}(k,z)$ napodobující $w(z)$
\item Kombinovanou modifikací jak gravitace tak EoS temné energie
\end{itemize}

Prolomení degenerací vyžaduje více nezávislých sond: rychlost růstu $f\sigma_8(z)$, slabá čočka, peculiární rychlosti.

\subsection{Závěry}

\begin{enumerate}
\item \textbf{Současný stav}: QCT s exploračními parametry ($X_0 = 10$, $z_{\rm structure} = 2$) je \textbf{falzifikováno Planck a DESI při $>10\sigma$}.

\item \textbf{Fyzikální diagnóza}: Fenomenologické parametry produkují $\sim 20\%$ odchylky v kosmologických pozorovatelných veličinách, daleko přesahující současnou přesnost ($\sim 1\%$).

\item \textbf{Cesta vpřed}: Explorace parametrického prostoru indikuje, že kompatibility je dosažitelné s:
\begin{align*}
X_0 &\sim 0.01\text{--}0.05 \quad (\text{redukce faktorem 200--1000}) \\
z_{\rm structure} &\sim 20\text{--}50 \quad (\text{zvýšení faktorem 10--25})
\end{align*}

\item \textbf{Vědecká hodnota}: Tato analýza demonstruje, že QCT je \textbf{falzifikovatelné}—kritický požadavek pro vědecké teorie. Rámec \textit{není} vyloučen, ale vyžaduje \textbf{datově řízenou optimalizaci parametrů}.

\item \textbf{Budoucí práce}:
\begin{itemize}
\item MCMC explorace parametrického prostoru $(X_0, z_{\rm structure}, \alpha)$
\item Bayesovský výběr modelu vs $\Lambda$CDM a alternativy modifikované gravitace
\item Mikroskopické odvození $w(z)$ z QCT polních rovnic
\item Modifikovaný Boltzmannův kód pro rigorózní CMB/BAO predikce
\end{itemize}
\end{enumerate}

\textbf{Verdikt}: Rámec QCT zůstává životaschopný, ale přechod od „kvalitativních predikcí" k „kvantitativnímu fitování dat" je zásadní pro publikaci v recenzovaných kosmologických časopisech. Tato příloha poskytuje statistické nástroje a plán pro tento přechod.

\vspace{1cm}
\begin{tcolorbox}[colback=yellow!10!white, colframe=orange!75!black, title=Poznámka k publikační strategii]
\textbf{Doporučení pro QCT rukopis:}

Vzhledem k významným napětím se současnými parametry navrhujeme:
\begin{enumerate}
\item \textbf{Prezentovat tuto přílohu} k demonstraci vědecké rigoróznosti a falzifikovatelnosti
\item \textbf{Explicitně uznat nejistoty parametrů} v Abstraktu a Závěrech
\item \textbf{Rámovat jako „rámec vyžadující optimalizaci"} spíše než „finální predikce"
\item \textbf{Zdůraznit testovatelnost} budoucími experimenty (CMB-S4, Euclid)
\end{enumerate}

Tento přístup ukazuje \textit{poctivé posouzení} stavu modelu—zásadní pro kredibilitu v kosmologické komunitě.
\end{tcolorbox}
