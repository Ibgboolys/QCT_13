% Appendix: Constraints on the entanglement scalar φ
\section{Constraints on the entanglement field $\varphi$ (fifth--force, atomic clocks)}
\label{app:phi}

We introduce a scalar field $\varphi$ with kinetics and a coupling to EM sectors via $f(\varphi)F^{2}$ (see main text). Here we summarize the order-bounds on the parameters from fifth-force and atomic clock experiments.

\subsection{Model and linearization}
We use the expansion $f(\varphi)=1+\beta_1(\varphi-\varphi_0)/M_*+\beta_2(\varphi-\varphi_0)^{2}/M_*^{2}+\cdots$, so
\begin{equation}
\frac{\delta\alpha_{\rm EM}}{\alpha_{\rm EM}} \simeq -\frac{\delta f}{f} \simeq -\beta_1\frac{\delta\varphi}{M_*}\,.
\end{equation}
A scalar field with potential $V(\varphi)=\tfrac12 m_\varphi^{2}(\varphi-\varphi_0)^{2}+\cdots$ has a Yukawa profile with range $\lambda=1/m_\varphi$ in the linear approximation.

\subsection{Fifth--force limits}
Laboratory and planetary inverse-square tests typically limit forces to a range of $\lambda\gtrsim\,$mm. To avoid direct limits for QCT:
\begin{itemize}
\item \textbf{Short range:} $m_\varphi\gtrsim 10^{-3}\,\mathrm{eV}$ ($\lambda\lesssim 0.2\,$mm) \Rightarrow the fifth force is suppressed by the range.
\item \textbf{Weak coupling:} If $m_\varphi\ll 10^{-3}\,$eV, we require $|\beta_1/M_*|\ll 10^{-24}\,\mathrm{GeV}^{-1}$ for the profiled potential deviations to be below the torsion–balance limits.
\end{itemize}
Both strategies are compatible with the cosmological use of $\varphi$ as a slow modulation: we prefer heavy $\varphi$ in laboratory conditions and/or screening.

\subsection{Atomic clocks and stability of $\alpha_{\rm EM}$}
A long-term comparison of optical clocks gives $|\dot\alpha_{\rm EM}/\alpha_{\rm EM}|\lesssim 10^{-17}\,\mathrm{yr}^{-1}$ (of the order of magnitude). Linearization yields
\begin{equation}
\left|\frac{\dot\alpha_{\rm EM}}{\alpha_{\rm EM}}\right| \simeq \left|\beta_1\frac{\dot\varphi}{M_*}\right| \lesssim 10^{-17}\,\mathrm{yr}^{-1}.
\end{equation}
This can be accomplished either (i) by freezing $\varphi$ at the minimum of $V(\varphi)$ (\,$\dot\varphi\approx 0$\,), or (ii) by a small effective coupling $|\beta_1/M_*|$ in the local environment. Around the limit $|\Delta\alpha_{\rm EM}/\alpha_{\rm EM}|\lesssim 10^{-7}$ for $\sim$2 Gyr then implies \begin{equation}
\left|\beta_1\frac{\Delta\varphi}{M_*}\right| \lesssim 10^{-7}\,.
\end{equation}

\subsection{Working region summary}
Conservative working region parameters (compatible with QCT goals):
\begin{itemize}
\item $m_\varphi\gtrsim 10^{-3}\,\mathrm{eV}$ (short range in the laboratory),
\item $|\beta_1/M_*|\lesssim 10^{-24}\,\mathrm{GeV}^{-1}$ (weak effective coupling to EM in the clock environment),
\item $\varphi$ close to the minimum of $V$ today ($\dot\varphi\approx 0$), so that $\dot\alpha_{\rm EM}/\alpha_{\rm EM}$ is negligible.
\end{itemize}
These conditions ensure agreement with torsion–balance, atomic clocks, and Oklo, while preserving the role of $\varphi$ as the bearer of energy accounting consistency ($\nabla_\mu T^{\mu\nu}_{\rm tot}=0$) and modulation in the early universe.