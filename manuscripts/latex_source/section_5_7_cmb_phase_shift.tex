\section{Cosmological Validation: The Neutrino Phase Transition}
\label{sec:cmb_phase_shift}

A critical test for any theory involving the cosmic neutrino background is the preservation of the acoustic peaks in the Cosmic Microwave Background (CMB). Standard Cosmology ($\Lambda$CDM) relies on the free-streaming nature of neutrinos at the epoch of recombination ($z \approx 1100$) to generate the observed phase shift in the acoustic spectrum \cite{Bashinsky2004}. Any model predicting a "stiff" neutrino fluid at this epoch is ruled out by Planck 2018 data \cite{PlanckCollaboration2020}.

QCT resolves this challenge naturally through a \textbf{thermodynamic phase transition mechanism}. Just as a superconductor loses its coherent electrical properties above a critical temperature $T_c$, the neutrino vacuum loses its coherent gravitational stiffness when the thermal energy exceeds the binding energy of the condensate pairs.

\subsection{Physical Mechanism: Thermal Decoherence}

At the photon decoupling epoch ($z \approx 1100$), the C$\nu$B temperature was $T_\nu \approx 0.17$ eV (approx. 1950 K). The thermal energy scale is significantly higher than the coherence energy required to maintain the macroscopic wavefunction $\Psi_{\nu\nu}$:
\begin{equation}
    k_B T_\nu(z=1100) \gg E_{\text{coh}}.
\end{equation}

In this high-temperature regime, the condensate undergoes a phase transition to a symmetric, disordered state. The macroscopic order parameter vanishes ($\langle \Psi \rangle \to 0$), and the "vacuum pressure" mechanism described in Section \ref{sec:emergent_gravity} turns off. Consequently, neutrinos behave as a standard relativistic gas of free particles.

\subsection{Quantitative Prediction: Interaction Rate Suppression}

Recent precise measurements of the phase shift in CMB acoustic oscillations induced by the C$\nu$B provide stringent constraints on neutrino self-interactions in the early universe~\cite{Montefalcone2025}. QCT neutrino pairing is fully consistent with these observations.

The effective interaction rate for QCT BCS-like pairing scales as:
\begin{equation}
\Gamma_{\rm QCT}(z) \sim \left(\frac{T_\nu(z)}{\Lambda_{\rm QCT}(z)}\right)^5 \times \frac{T_\nu(z)}{\hbar},
\label{eq:gamma_qct_z}
\end{equation}
where $T_\nu(z) = T_{\rm CMB,0}(1+z)$ is the neutrino temperature and $\Lambda_{\rm QCT}(z) = (3/2)\sqrt{E_{\rm pair}(z) \times m_p}$ is the running cutoff scale.

The steep $T^5$ temperature dependence arises from the BCS pairing mechanism mediated by a heavy boson of mass $\Lambda_{\rm QCT} \sim 100$~TeV, well above neutrino temperatures throughout cosmological history. This scaling is characteristic of self-interactions via heavy mediators~\cite{CyrRacine2014, Oldengott2017, Kreisch2020}.

For the logarithmic evolution $E_{\rm pair}(z) = E_0 + \kappa_{\rm conf}\ln(1+z)$ (Eq.~\ref{eq:E_pair_evolution}), the cutoff grows only logarithmically with redshift:
\begin{equation}
\Lambda_{\rm QCT}(z) = \frac{3}{2}\sqrt{[E_0 + \kappa_{\rm conf}\ln(1+z)] \times m_p}.
\end{equation}

This slow growth, combined with the steep $T^5$ dependence, ensures $\Gamma_{\rm QCT} \ll H(z)$ throughout the cosmologically relevant epoch $z < 10^{12}$.

At the CMB constraint redshift $z \sim 1.7 \times 10^4$~\cite{Montefalcone2025}:
\begin{align}
T_\nu &\approx 3.1~{\rm eV}, \quad \Lambda_{\rm QCT} \approx 98~{\rm TeV}, \\
\frac{T_\nu}{\Lambda_{\rm QCT}} &\sim 3.2 \times 10^{-14}, \quad
\left(\frac{T_\nu}{\Lambda_{\rm QCT}}\right)^5 \sim 3.2 \times 10^{-68}, \\
\frac{\Gamma_{\rm QCT}}{H} &\sim 1.2 \times 10^{-27} \ll 1.
\end{align}

The extremely small coupling strength $(T/\Lambda)^5 \sim 10^{-68}$ renders QCT interactions completely negligible for CMB observables. Even at much higher redshifts ($z \sim 10^9$, BBN epoch), we find $\Gamma_{\rm QCT}/H \sim 10^{-13} \ll 1$, confirming free-streaming throughout radiation domination.

This results in a phase-shift amplitude ratio:
\begin{equation}
\mathcal{A}_\infty^{\rm QCT} = 1.00,
\end{equation}
identical to the Standard Model free-streaming prediction, in perfect agreement with CMB measurements: $\mathcal{A}_\infty > 0.90$ at 95\% confidence level~\cite{Montefalcone2025}.

\subsection{Implications for $E_{\rm pair}(z)$ Evolution}

The CMB constraint provides an independent validation of the logarithmic form of $E_{\rm pair}(z)$ over alternative evolution models. Consider a hypothetical conformal evolution $E_{\rm pair} \propto \Omega^2 \propto (1+z)^2$ (as might naively be expected from conformal scaling). This would yield:
\begin{equation}
\Lambda_{\rm QCT}(z) \propto \sqrt{E_{\rm pair}(z)} \propto (1+z),
\end{equation}
causing the dimensionless coupling to scale as:
\begin{equation}
\frac{T_\nu}{\Lambda_{\rm QCT}} \propto \frac{(1+z)}{(1+z)} = \text{const}.
\end{equation}

The interaction rate would then scale as:
\begin{equation}
\frac{\Gamma_{\rm QCT}}{H} \sim \left(\frac{T_\nu}{\Lambda_{\rm QCT}}\right)^5 \times \frac{T_\nu}{H} \propto (1+z)^{-1}.
\end{equation}

For $\Gamma_{\rm QCT}/H \sim 10^{-27}$ at $z \sim 10^4$, this scaling would give $\Gamma/H \sim 10^{-16}$ at $z \sim 10^{15}$ (electroweak scale), potentially leading to late decoupling that would suppress the CMB phase shift below observed levels.

\textbf{In contrast}, the logarithmic form ensures $\Lambda_{\rm QCT}$ increases slowly ($\propto \sqrt{\ln(1+z)}$), maintaining $\Gamma \ll H$ even at high $z$, consistent with CMB data. This provides independent observational support for the logarithmic evolution derived from the confinement mechanism (Section~\ref{sec:cosmological_confinement}).

\subsection{Null-Test Validation of QCT}

QCT predicts \textit{no deviation} from Standard Model neutrino free-streaming in CMB observables—a prediction known as a \textit{null test}. The CMB measurements confirming $\mathcal{A}_\infty \approx 1.00$ constitute a successful validation of this prediction.

This null test is non-trivial for several reasons:
\begin{enumerate}
\item \textbf{No fine-tuning required.} The QCT cutoff scale $\Lambda_{\rm QCT} \sim 100$~TeV emerges naturally from the pairing mechanism (Eq.~\ref{eq:lambda_qct_derivation}) and the muon $g-2$ constraint (Section~\ref{sec:muon_g2}). This scale is sufficiently large that interactions remain negligible during radiation domination without any parameter adjustment.

\item \textbf{Consistency across scales.} QCT must simultaneously explain:
\begin{itemize}
\item Gravitational screening at sub-mm scales ($\lambda_{\rm screen} \sim 1$~mm)
\item Modified gravity at astrophysical scales ($G_{\rm eff} = 0.9\,G_N$)
\item Free-streaming neutrinos at cosmological scales (CMB)
\end{itemize}
The fact that the same framework with the same parameters satisfies all three constraints strengthens the theory's consistency.

\item \textbf{Indirect validation of microscopic physics.} The CMB constraint tests the fundamental QCT interaction mechanism without directly measuring it, similar to how Big Bang Nucleosynthesis constrains particle physics at MeV scales. The agreement validates not just the phenomenology but the underlying BCS-like pairing structure.
\end{enumerate}

\subsection{Comparison with Alternative Neutrino Interaction Scenarios}

For context, we compare QCT with other neutrino interaction models constrained by CMB phase-shift measurements:

\begin{table}[H]
\centering
\caption{Comparison of neutrino interaction models and CMB constraints.}
\label{tab:cmb_neutrino_models}
\begin{tabular}{lccc}
\hline
Model & $\Gamma \propto$ & $z_{\rm dec}$ constraint & $\mathcal{A}_\infty$ \\
\hline
SM (free-streaming) & 0 & $z \to \infty$ & 1.00 \\
QCT (BCS pairing) & $T^5/\Lambda_{\rm QCT}^4$ & $\gg 10^{12}$ & 1.00 \\
Neutrino SI (heavy med.) & $T^5$ & $> 1.7 \times 10^4$ & $> 0.90$ \\
Neutrino-DM scattering & $T^3$ & $> 1.3 \times 10^4$ & $> 0.90$ \\
\hline
\end{tabular}
\end{table}

\noindent QCT falls into the same phenomenological category as neutrino self-interactions (SI) via heavy mediators, with $\Gamma \propto T^5$ scaling. However, the much larger mediator scale ($\Lambda_{\rm QCT} \sim 100$~TeV vs typical SI models with $\Lambda \sim 1$~TeV) pushes the decoupling redshift far beyond the CMB sensitivity range.

\subsection{Late-Time "Freezing": Emergence of Gravitational Effects}

The emergent gravitational effects of QCT (dark matter mimicry at galactic scales, Section~\ref{sec:galactic_rotation}) appear only as a "freezing" phenomenon in the late, cold universe ($z \ll 100$), analogous to the formation of ice on a cooling structure. This ensures:
\begin{itemize}
\item \textbf{Early universe ($z > 1100$):} Free-streaming neutrinos → CMB acoustic peaks preserved
\item \textbf{Late universe ($z < 100$):} Condensate coherence → Modified gravity at galactic scales
\end{itemize}

This phase transition naturally reconciles QCT with precision CMB measurements without fine-tuning. The same cutoff scale $\Lambda_{\rm QCT} \sim 100$~TeV that explains the muon $g-2$ anomaly (Section~\ref{sec:muon_g2}) ensures negligible interactions during radiation domination.

\subsection{Future Prospects}

Upcoming CMB experiments will test this prediction with improved precision:
\begin{itemize}
\item \textbf{Simons Observatory} (operational $\sim$2027): Expected precision $\delta \mathcal{A}_\infty \sim 0.01$. QCT predicts $\mathcal{A}_\infty = 1.000 \pm 0.000$, remaining consistent.
\item \textbf{CMB-S4}: Will reach $\delta \mathcal{A}_\infty \sim 0.001$, providing a null-test precision of 0.1\%.
\item \textbf{Large-Scale Structure}: The same phase shift appears in baryon acoustic oscillations (BAO)~\cite{Baumann2017_BAO}. At late times ($z < 2$), after condensate formation, QCT predicts modified phase shift $\beta_\phi \sim 1.4$ due to $G_{\rm eff} = 0.9\,G_N$ and non-adiabatic perturbations, compatible with recent DESI measurements (Section~\ref{sec:bao_consistency}).
\end{itemize}

The key discriminator for QCT will be in \textit{different} observables: modified growth rate $f(z)\sigma_8$ due to $G_{\rm eff} = 0.9\,G_N$, sub-mm gravity tests, and late-time BAO phase shifts.
