\section{Phase Coherence Near Black Holes: Resolution of Apparent Paradox}
\label{app:bh_coherence}

\subsection{Motivation: the stiffness paradox}

Our analysis of phase coherence requirements (Sec.~\ref{sec:epair_decoherence}) revealed an apparent tension: if the coherence length $\xi$ were set by the Schwarzschild radius $r_S$, achieving $\sigma^2_{\rm avg} \sim 1$ would require:
\begin{equation}
f^2_{\rm required}(r_S) \sim \frac{\xi}{4\pi^2 V_{\rm proj}} \cdot \frac{1}{c_s}
\label{eq:f2_rs_scaling}
\end{equation}
For stellar-mass and supermassive black holes, this yields:
\begin{align}
\text{Sun: } & r_S = 2.95\,\text{km} \quad \Rightarrow \quad f^2 \sim 10^{22}\,\text{m}^{-3}\cdot\text{eV} \\
\text{M87*: } & r_S = 1.2 \times 10^{13}\,\text{m} \quad \Rightarrow \quad f^2 \sim 10^{32}\,\text{m}^{-3}\cdot\text{eV}
\end{align}
These values exceed the standard condensate stiffness ($f^2 \sim 3 \times 10^{15}$) by $7\text{--}17$ orders of magnitude, suggesting unphysically \emph{rigid} condensates near larger black holes—contrary to intuitive expectations that extreme gravitational fields should \emph{disrupt} quantum coherence.

\subsection{Resolution: universal coherence length}

\noindent\textbf{Key insight.} The coherence length $\xi$ is \emph{not} determined by spacetime curvature scales but by \emph{intrinsic} condensate properties:
\begin{equation}
\xi \sim \lambda_{\rm screen} = \frac{R_{\rm proj}}{\ln(1/f_{\rm screen})} \approx 1.0\,\text{mm} \quad \text{(universal)}
\label{eq:xi_universal}
\end{equation}
This follows from the screening relation Eq.~\eqref{eq:screening_relation}, which depends on the mass ratio $m_\nu/m_p \approx 10^{-10}$ and local baryon density, \emph{not} on $r_S$.

\subsection{Physical implications}

\subsubsection{Condensate behavior near horizons}

The neutrino condensate exhibits universal coherence $\xi \sim 1\,\text{mm}$ regardless of ambient curvature. Near a black hole horizon:
\begin{itemize}
    \item \textbf{Metric effects:} Schwarzschild geometry modifies $g_{\mu\nu}$, affecting geodesic motion and gravitational redshift.
    \item \textbf{Condensate structure:} Phase stiffness $f^{2}$ remains $\sim 3 \times 10^{15}\,\text{m}^{-3}\cdot\text{eV}$, controlled by local interactions (weak scattering, thermal bath).
    \item \textbf{Pairing energy:} $E_{\rm pair} = 5.38 \times 10^{18}\,\text{eV}$ is a cosmological quantity (confinement mechanism) independent of local curvature.
    \item \textbf{Screening:} Gravitational coupling suppressed by $\exp(-r_S/\xi)$ only if $r_S \lesssim \xi$.
\end{itemize}

\subsection{Resolution with phase decoherence saturation}

Naively, for astrophysical black holes with $r_S \gg \xi \sim 1\,\text{mm}$, the exponential screening would give $G_{\rm eff} \sim \exp(-r_S/\xi) \approx 0$, catastrophically suppressing gravity. However, this is prevented by \emph{phase variance saturation} (see Appendix~\ref{app:kernel_eft}, Eq.~\ref{eq:sigma_squared_saturation}).

\paragraph{Corrected formula for $G_{\rm eff}$.}

The full effective gravity includes both Yukawa screening (sub-mm) and phase decoherence (all scales):

\begin{equation}
G_{\rm eff}(r) = G_N \times \underbrace{\min\left[e^{-r/\lambda_{\rm screen}}, 1\right]}_{\text{Yukawa (sub-mm)}} \times \underbrace{\exp\left(-\frac{\sigma^2(r)}{2}\right)}_{\text{phase decoherence}}
\end{equation}

\noindent For $r \gg R_{\rm proj} \approx 2.3\,\text{cm}$ (all astrophysical scales):
\begin{equation}
\sigma^2(r) \to \sigma_{\max}^2 \approx 0.2 \quad \Rightarrow \quad G_{\rm eff} \to 0.9\, G_N
\end{equation}

\paragraph{Black hole observables.}

\textbf{Schwarzschild radius:} Unchanged by QCT (definition).

\textbf{Photon sphere radius:}
\begin{equation}
r_{\rm ph}^{\rm QCT} = \frac{3GM}{c^2} \times \frac{1}{G_{\rm eff}/G_N} \approx 1.11 \times r_{\rm ph}^{\rm GR}
\end{equation}

\textbf{Shadow radius (observer at infinity):}
\begin{equation}
r_{\rm shadow}^{\rm QCT} = \sqrt{\frac{27 G_{\rm eff} M}{c^2}} \approx \sqrt{0.9} \times r_{\rm shadow}^{\rm GR} \approx 0.95 \times r_{\rm shadow}^{\rm GR}
\end{equation}

\textbf{Event Horizon Telescope constraints:} For M87* ($M = 6.5 \times 10^9 M_\odot$, distance $D = 16.8\,\text{Mpc}$):
\begin{align}
r_{\rm shadow}^{\rm GR} &= \sqrt{27} \times \frac{GM}{c^2} \approx 2.6 \times r_S \\
\theta_{\rm shadow}^{\rm GR} &= \frac{r_{\rm shadow}}{D} \approx 42\,\mu\text{as} \\
\theta_{\rm shadow}^{\rm QCT} &\approx 0.95 \times 42 \approx 40\,\mu\text{as}
\end{align}

\noindent Current EHT measurement: $\theta_{\rm obs} = 42 \pm 3\,\mu\text{as}$. QCT prediction is within $1\sigma$ uncertainty. Future EHT improvements (higher resolution, more telescopes) will constrain this further.

\paragraph{Quasi-normal modes and gravitational waves.}

For Schwarzschild black holes, fundamental QNM frequency:
\begin{equation}
f_{\rm QNM} = \frac{c^3}{2\pi GM} \times \underbrace{0.3737}_{\text{dimensionless}} \times \sqrt{\frac{G_{\rm eff}}{G_N}}
\end{equation}

\noindent For $G_{\rm eff} = 0.9\, G_N$:
\begin{equation}
f_{\rm QNM}^{\rm QCT} \approx 0.95 \times f_{\rm QNM}^{\rm GR}
\end{equation}

\noindent This $5\%$ shift is potentially measurable with LIGO/Virgo for nearby binary black hole mergers with high SNR. Future detectors (Einstein Telescope, Cosmic Explorer, LISA) will provide stringent tests.

\paragraph{Accretion disk dynamics.}

Innermost stable circular orbit (ISCO):
\begin{equation}
r_{\rm ISCO}^{\rm QCT} = \frac{6GM}{c^2} \times \frac{1}{G_{\rm eff}/G_N} \approx 1.11 \times r_{\rm ISCO}^{\rm GR}
\end{equation}

\noindent This affects:
\begin{itemize}
  \item X-ray spectral fits (inner disk temperature $T_{\rm in} \propto r_{\rm ISCO}^{-3/4}$)
  \item Iron K$\alpha$ line profiles (relativistic broadening)
  \item Quasi-periodic oscillations (QPOs) in accreting systems
\end{itemize}

\paragraph{Primordial black holes (PBHs).}

For PBHs with $M \lesssim M_\oplus$, QCT effects are negligible on astrophysical observables but crucial for sub-mm Hawking radiation spectrum modifications (beyond scope of this work).

\subsection{Summary}

The apparent paradox is resolved by phase decoherence saturation:
\begin{enumerate}
    \item Coherence length $\xi \sim 1\,\text{mm}$ is \emph{universal}, set by condensate microphysics.
    \item Phase variance $\sigma^2(r)$ saturates at $\sigma_{\max}^2 \approx 0.2$ for $r \gg R_{\rm proj}$.
    \item Effective gravity approaches $G_{\rm eff} \approx 0.9\, G_N$ on all astrophysical scales, \emph{not zero}.
    \item QCT predicts $\sim 5\%$ corrections to black hole observables (shadows, QNMs, ISCO).
\end{enumerate}
This is testable via Event Horizon Telescope, LIGO/Virgo ringdown analysis, and X-ray observations of accreting black holes.

\subsection{Connection to Painlevé-Gullstrand Formalism}
\label{app:bh_painleve_gullstrand}

The resolution of the black hole paradox (Sec.~\ref{app:bh_coherence}) via phase decoherence saturation can be understood more rigorously through the Painlevé-Gullstrand (PG) formulation of analogue gravity~\cite{Hossenfelder2020, Barcelo2005}. This connection establishes QCT within the established framework of acoustic black hole analogues.

\subsubsection{Schwarzschild in Painlevé-Gullstrand Coordinates}

\paragraph{Standard Schwarzschild metric.}

For a static, spherically symmetric black hole in $n+1$ dimensions:
\begin{equation}
ds^2 = -\gamma(r)dt^2 + \frac{dr^2}{\gamma(r)} + r^2 d\Omega^2_{n-1}, \quad \gamma(r) = 1 - \frac{2GM}{r},
\label{eq:schwarzschild_standard}
\end{equation}
where $d\Omega^2_{n-1}$ is the metric on a $(n-1)$-dimensional sphere and $M$ is the black hole mass.

\paragraph{Painlevé-Gullstrand transformation.}

Following~\cite{Painleve1921, Gullstrand1922, Hossenfelder2020}, introduce new time coordinate $t'$ such that:
\begin{equation}
dt = dt' - \frac{\sqrt{1-\gamma(r)}}{\gamma(r)} dr.
\end{equation}

The metric becomes:
\begin{equation}
ds^2 = -\kappa^2\gamma(r)dt'^2 + 2\kappa\sqrt{1-\gamma(r)} dt'dr + dr^2 + r^2 d\Omega^2_{n-1},
\label{eq:schwarzschild_PG}
\end{equation}
where $\kappa$ is a normalization constant (chosen for convenience, typically $\kappa = 1$).

\paragraph{Acoustic Metric Identification.}

From the non-relativistic acoustic metric (Hossenfelder Eq.~11-12):
\begin{equation}
g^{\mu\nu}_{\text{acoustic}} \propto \left(\frac{\rho_0}{c}\right)^{-2/(n-1)} \begin{pmatrix}
-1/c^2 & -v^j_0/c^2 \\
-v^i_0/c^2 & \delta^{ij} - v^i_0 v^j_0/c^2
\end{pmatrix},
\end{equation}

Comparing with the PG metric (Eq.~\ref{eq:schwarzschild_PG}), one reads off:
\begin{align}
c_0 &= \kappa, \\
\rho_0 &= \kappa \Omega(r)^{n-1}, \\
v^r_0 &= \kappa\sqrt{1-\gamma(r)}, \\
v^\theta_0 &= v^\phi_0 = 0 \quad \text{(spherical symmetry)}.
\end{align}

where $\Omega(r)$ is a conformal factor to be determined by the fluid equations.

\subsubsection{QCT Modification}

\paragraph{Environment-dependent gravity.}

In QCT, the gravitational constant depends on environment:
\begin{equation}
G \to G_{\text{eff}}(r) = G_N \times \min\left[e^{-r/\lambda_{\text{screen}}(r)}, 1\right] \times \exp\left(-\frac{\sigma^2(r)}{2}\right).
\end{equation}

This modifies the blackening function:
\begin{equation}
\gamma_{\text{QCT}}(r) = 1 - \frac{2G_{\text{eff}}(r) M}{r}.
\label{eq:gamma_QCT}
\end{equation}

\paragraph{Neutrino density at horizon.}

The neutrino condensate accumulates in the gravitational well:
\begin{equation}
n_\nu(r) = n_{\nu,0} \cdot K(r), \quad K(r) = 1 + \alpha\frac{\Phi(r)}{c^2} = 1 - \alpha\frac{GM}{r c^2},
\end{equation}
with $\alpha \approx -9 \times 10^{11}$.

For a stellar-mass black hole ($M = M_\odot$, $r_S = 2.95$ km):
\begin{align}
K(r_S) &= 1 + 9 \times 10^{11} \times \frac{1.48 \times 10^3}{9 \times 10^{16}} \approx 1.5 \times 10^{28}, \\
\xi(r_S) &= \frac{\xi_0}{\sqrt{K(r_S)}} \sim \frac{1 \text{ mm}}{1.2 \times 10^{14}} \sim 8 \times 10^{-18} \text{ m} \quad \text{(extreme decoherence)}.
\end{align}

\paragraph{QCT conformal factor.}

The QCT conformal factor (from Sec.~\ref{sec:screening_conformal}):
\begin{equation}
\Omega_{\text{QCT}}(r) = \sqrt{f_{\text{screen}} \cdot K(r)} = \sqrt{\frac{m_\nu}{m_p}} \cdot \sqrt{1 + \alpha\frac{\Phi(r)}{c^2}}.
\end{equation}

\paragraph{Comparison: Hossenfelder vs. QCT.}

For Schwarzschild, Hossenfelder derives (their Eq.~33):
\begin{equation}
\Omega_{\text{Hossenfelder}}(r) = \frac{1}{r}\left[1-\gamma(r)\right]^{1/(n-1)}, \quad (n=3).
\end{equation}

\textbf{Key difference:}
\begin{itemize}
\item \textbf{Hossenfelder:} $\Omega(r) \to \infty$ at $r = r_S$ (horizon). This is acceptable for classical fluid analogue, where $\rho_0 = \kappa \Omega^{n-1} \to \infty$ simply means infinite fluid density at horizon.

\item \textbf{QCT:} $\Omega_{\text{QCT}}(r_S)$ remains finite due to phase decoherence saturation. The neutrino density $n_\nu(r_S) \sim 10^{28} n_{\nu,0}$ is large but not divergent. Coherence length $\xi(r_S) \sim 10^{-18}$ m is extremely short, but the saturation mechanism prevents $G_{\text{eff}} \to 0$.
\end{itemize}

\subsubsection{Physical Consequences}

\paragraph{Modified horizon structure.}

The QCT horizon radius is modified:
\begin{equation}
r_{S,\text{QCT}} : \quad \gamma_{\text{QCT}}(r_{S,\text{QCT}}) = 0 \quad \Rightarrow \quad r_{S,\text{QCT}} = 2G_{\text{eff}}(r_{S,\text{QCT}}) M.
\end{equation}

For astrophysical scales where $G_{\text{eff}} \approx 0.9 G_N$ (saturation regime):
\begin{equation}
r_{S,\text{QCT}} \approx 0.9 \times r_{S,\text{GR}}.
\end{equation}

\paragraph{Photon sphere and shadow.}

From Appendix~\ref{app:bh_coherence}, observable quantities:
\begin{align}
r_{\text{ph}}^{\text{QCT}} &\approx 1.11 \times r_{\text{ph}}^{\text{GR}}, \\
r_{\text{shadow}}^{\text{QCT}} &\approx 0.95 \times r_{\text{shadow}}^{\text{GR}}.
\end{align}

The Painlevé-Gullstrand formalism shows these arise from modifications to $\gamma(r)$ (Eq.~\ref{eq:gamma_QCT}), which in turn modify fluid velocity:
\begin{equation}
v^r_0 = \kappa\sqrt{1-\gamma_{\text{QCT}}(r)} \approx \kappa\sqrt{\frac{2G_{\text{eff}}(r) M}{r}}.
\end{equation}

\subsubsection{Summary}

The Painlevé-Gullstrand formulation establishes the rigorous connection:

\begin{equation}
\boxed{
\begin{aligned}
&\textbf{Hossenfelder (classical):} \quad \Omega(r) = \frac{1}{r}[1-\gamma(r)]^{1/(n-1)} \quad \text{(from continuity eqn.)} \\
&\textbf{QCT (quantum):} \quad \Omega_{\text{QCT}}(r) = \sqrt{f_{\text{screen}} K(r)} \quad \text{(from condensate coherence)}
\end{aligned}
}
\end{equation}

Both satisfy the fluid equations, but QCT's quantum origin:
\begin{itemize}
\item Predicts saturation: $\Omega_{\text{QCT}}$ finite at $r = r_S$ (no divergence)
\item Environment-dependent: $K(r) = 1 + \alpha\Phi(r)/c^2$ (testable via ISS vs. Earth)
\item Astrophysical predictions: $r_{\text{shadow}}^{\text{QCT}} \approx 0.95 \times r_{\text{shadow}}^{\text{GR}}$ (EHT constraint)
\end{itemize}

This completes the analogue gravity foundation for QCT black hole physics, resolving the apparent paradox via the interplay of conformal rescaling and phase decoherence saturation.