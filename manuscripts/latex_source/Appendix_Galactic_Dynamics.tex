\section{Appendix P: Microscopic Derivation of Galactic Dynamics}
\label{app:galactic_dynamics}

In this appendix, we detail the transition from the microscopic QCT metric to the effective galactic dynamics used in Section \ref{sec:galactic_rotation}.

\subsection{Effective Metric and Vacuum Flow}
Starting from the modified Painlevé-Gullstrand metric (derived in Eq. 831 of the main theory), the radial fluid velocity of the space-time flow induced by the condensate is given by:
\begin{equation}
    v_{flow}(r) = c \sqrt{1 - \gamma_{QCT}(r)}.
\end{equation}
In the weak-field regime characteristic of galactic outskirts, the coherence condition of the neutrino condensate imposes a non-linear scaling on the effective coupling. The vacuum contribution to the potential $\Phi_{vac}$ scales as $\sqrt{M_{bar}}$, leading to an emergent acceleration term.

\subsection{The Emergent Velocity Law}
The total effective velocity $V_{QCT}$ observed in the galactic plane is the quadratic sum of the baryonic velocity (Newtonian) and the vacuum flow velocity:
\begin{equation}
    V_{QCT}^2(r) = V_{bar}^2(r) + V_{vac}^2(r).
\end{equation}
Substituting the condensate response function, the vacuum term takes the specific form:
\begin{equation}
    V_{vac}^2(r) = \sqrt{G_N M_{bar}(<r) a_0},
\end{equation}
where $a_0$ is the critical acceleration scale. This algebraic relation is mathematically equivalent to the "Rarefaction Limit" of MOND but arises here from the saturation of the phase coherence length $\xi$.

\subsection{Simulation Parameters}
The validation presented in Fig. \ref{fig:galaxy_sim} utilizes the observed baryonic mass models (stellar disk + HI gas) from the SPARC database \cite{Lelli2016}. To ensure a rigorous test without overfitting, we adhered to the following constraints:

\begin{itemize}
    \item \textbf{Stellar Mass-to-Light Ratio:} Fixed at $\Upsilon_* = 0.5 \, M_{\odot}/L_{\odot}$ for the $3.6 \mu m$ band, consistent with population synthesis models.
    \item \textbf{Critical Acceleration:} Fixed globally at $a_0 = 3700 \, (\text{km/s})^2/\text{kpc} \approx 1.2 \times 10^{-10} \, \text{m/s}^2$.
    \item \textbf{No Free Parameters:} No galaxy-specific parameter tuning (nuisance parameters) was performed. The fit is a direct prediction based solely on the observed distribution of baryons.
\end{itemize}

The exceptional agreement for the gas-dominated galaxy NGC 1560 (error $< 5\%$) confirms that the vacuum response term $\sqrt{G M a_0}$ correctly accounts for the missing mass without requiring non-baryonic dark matter particles.
