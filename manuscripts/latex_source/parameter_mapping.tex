% Tabulka mapování mikroskopických parametrů na EFT
% Pro integraci do preprintu

\section{Parameter Mapping: Microscopic Derivation ↔ EFT}
\label{sec:parameter_mapping}

This section provides explicit conversion relations between the parameters of the microscopic derivation (based on the neutrino condensate $\Psi_{\nu\nu}$) and the effective coefficients of the EFT framework.

\subsection{Basic Inputs: Cosmological Parameters}

\begin{table}[h]
\centering
\caption{Cosmological Inputs from C$\nu$B.}
\begin{tabular}{llll}
\toprule
\textbf{Parameter} & \textbf{Symbol} & \textbf{Value} & \textbf{Units} \\
\midrule
Relic Neutrino Density & $n_{\nu,{\rm phys}}$ & $336$ & cm$^{-3}$ \\
Neutrino Mass (Mean) & $m_\nu$ & $\sim 0.1$ & eV/$c^2$ \\
Quantum Compression Constant & $F_{\rm proj}$ & $2.43\times 10^4$ & (dimensionless) \\
Projection Volume (Derived) & $V_{\rm proj}$ & $72.3$ & cm$^3$ \\
Projection radius & $R_{\rm proj}$ & $2.6$ & cm \\
\bottomrule
\end{tabular}
\end{table}

\noindent\textbf{Derivation:}
\begin{equation}
V_{\rm proj} = \frac{F_{\rm proj}}{n_{\nu,{\rm phys}}} = \frac{2.43\times 10^4}{336\,{\rm cm}^{-3}} \approx 72.3\,{\rm cm}^3,
\end{equation}
\begin{equation}
R_{\rm proj} = \left(\frac{3V_{\rm proj}}{4\pi}\right)^{1/3} \approx 2.6\,{\rm cm}.
\end{equation}

\subsection{Binding Energy and Entanglement Density}

\paragraph{Neutrino Confinement.}
The $\nu\bar\nu$ pairs created in the early universe are "confined" by the cosmological expansion. The binding energy increases with the expansion factor:
\begin{equation}\label{eq:E_pair}
E_{\rm pair} \sim 10^{20} \times m_\nu c^2 \approx 10^{19}\,{\rm eV}.
\end{equation}
This is the amplification factor needed to reproduce the observed Newton's constant.

\paragraph{Effective entanglement density.}
\begin{equation}\label{eq:rho_ent_eff}
\rho_{\rm ent} = n_\nu \cdot E_{\rm pair} \approx (336\,{\rm cm}^{-3}) \times (10^{19}\,{\rm eV}) \approx 6\times 10^{-9}\,{\rm kg/m}^3.
\end{equation}
In natural units ($\hbar=c=1$):
\begin{equation}
\rho_{\rm ent} \approx 3.4\times 10^{-6}\,{\rm eV}^4 = 3.4\times 10^{-42}\,{\rm GeV}^4.
\end{equation}

\subsection{Gravity constant}

\paragraph{Microscopic relation.}
From the geometric summation of projection volumes:
\begin{equation}\label{eq:G_from_micro}
G_{\rm eff} = \alpha_G \cdot \frac{\rho_{\rm ent}\,V_{\rm proj}}{R_{\rm proj}},
\end{equation}
where $\alpha_G\approx 1$ is a dimensionless numerical factor (of the order of units).

\paragraph{Numerical verification.}
Substituting the values ​​from \eqref{eq:rho_ent_eff}:
\begin{align}
G_{\rm eff} &= 1 \times \frac{(6\times 10^{-9}\,{\rm kg/m}^3)(72.3\times 10^{-6}\,{\rm m}^3)}{2.6\times 10^{-2}\,{\rm m}} \\
&\approx 1.67\times 10^{-11}\,{\rm m}^3{\rm kg}^{-1}{\rm s}^{-2}.
\end{align}
With $\alpha_G\approx 4$ we get the exactly measured value $G=6.67\times 10^{-11}$.

\paragraph{Relation to EFT parameter.}
In the preprint (section 2, lagrangian.md) it appears:
\begin{equation}
G_{\rm eff} = \frac{G_0}{1+\kappa_{\rm grav}\cdot\rho_{\rm ent}/c^2}.
\end{equation}
In the limit $\kappa_{\rm grav}\cdot\rho_{\rm ent}\ll c^2$ we linearize:
\begin{equation}
G_{\rm eff} \approx G_0\left(1-\frac{\kappa_{\rm grav}\rho_{\rm ent}}{c^2}\right).
\end{equation}
For another regime (where $\rho_{\rm ent}$ dominates):
\begin{equation}
G_{\rm eff} \sim \frac{c^2}{\kappa_{\rm grav}\rho_{\rm ent}}.
\end{equation}
Compared to \eqref{eq:G_from_micro}:
\begin{equation}
\kappa_{\rm grav} \sim \frac{c^2\,R_{\rm proj}}{\alpha_G\,V_{\rm proj}\,\rho_{\rm ent}}.
\end{equation}
Numerically: $\kappa_{\rm grav}\sim 10^{-37}\,{\rm m}^3/{\rm kg}$ (consistent with lagrangian.md).

\subsection{Self-interaction: Quartic coupling λ}

\paragraph{Microscopic.}
The Schrödinger equation for a condensate \eqref{eq:schrodinger_cond} contains:
\begin{equation}
V_{\rm int} = g|\Psi_{\nu\nu}|^4.
\end{equation}
The coefficient $g$ has the dimension $[g]={\rm eV}\cdot{\rm m}^3$ (in SI).

\paragraph{EFT.}
The preprint (section 2) introduces:
\begin{equation}
V(|\Psi|) = \frac{\lambda}{4}(|\Psi|^2)^2,
\end{equation}
where $\lambda$ is dimensionless. In natural units ($\hbar=c=1$) we have $[\Psi]={\rm GeV}$, so $[\lambda]=1$ (dimensionless).

\paragraph{Conversion.}
In the macroscopic limit:
\begin{equation}
\Psi(\mathbf x) = \langle\Psi_{\nu\nu}\rangle_{V_{\rm proj}},
\end{equation}
so
\begin{equation}
|\Psi|^2 \sim \frac{1}{V_{\rm proj}}\int_{V_{\rm proj}}|\Psi_{\nu\nu}|^2\,d^3x'.
\end{equation}
From which:
\begin{equation}
\lambda \sim \frac{g}{V_{\rm proj}^2}\times(\text{conversion factors}).
\end{equation}
Estimate for weak interaction: $\lambda\sim 10^{-31}$ (consistent with core\_fields.md and parameter\_calibration.md).

\subsection{Cutoff scale Λ\_QCT}

\paragraph{Microscopic scale.}
The binding energy $E_{\rm pair}\sim 10^{19}\,{\rm eV}$ determines the energy scale above which the microscopic description ceases to apply (the pair "breaks up"). This gives:
\begin{equation}
\Lambda_{\rm micro} \sim E_{\rm pair} \sim 10^{19}\,{\rm eV} = 10^{10}\,{\rm GeV}.
\end{equation}

\paragraph{EFT scale.}
The preprint works with $\Lambda_{\rm QCT}\approx
107\,{\rm TeV} = 1.07\times 10^5\,{\rm GeV}$, which is the scale of new physics (non-perturbative RG jumps).

\paragraph{Explanation of the difference.}
$\Lambda_{\rm micro}$ is the microscopic "Planck scale" of the condensate (pair-breaking energy). $\Lambda_{\rm QCT}$ is the effective scale below which the EFT expansion is valid — it is the scale of the first large NP transition in the run $\alpha(\mu)$. The relation:
\begin{equation}
\Lambda_{\rm QCT} \sim \frac{\Lambda_{\rm micro}}{\sqrt{E_{\rm pair}/m_\nu}} \sim \frac{10^{10}\,{\rm GeV}}{10^{10}} \sim 1\,{\rm GeV}?
\end{equation}
The exact relation requires RG analysis (section NP--RG), but on the order of: $\Lambda_{\rm QCT}\ll\Lambda_{\rm micro}$.

\subsection{Speed ​​of light c and stiffness of condensate}

\paragraph{Microscopic relation.}
Velocity of excitations (photons):
\begin{equation}\label{eq:c_stiffness}
c^2 = \frac{K_{\rm cond}}{\rho_{\rm ent}},
\end{equation}
where $K_{\rm cond}$ is the bulk modulus ("stiffness") of the condensate.

\paragraph{Numerics.}
Substituting $c=3\times 10^8\,{\rm m/s}$ and $\rho_{\rm ent}\approx 6\times 10^{-9}\,{\rm kg/m}^3$:
\begin{equation}
K_{\rm cond} = \rho_{\rm ent}\,c^2 = (6\times 10^{-9})(9\times 10^{16}) \approx 5.4\times 10^8\,{\rm Pa}.
\end{equation}
That is a pressure of the order of **gigapascals** — reasonable for a quantum fluid in a confined state!

\paragraph{Relation to the α run.}

In the preprint, the run $\alpha(\mu)$ is governed by the NP--RG ansatz. Microscopically:
\begin{equation}
\alpha(\mu) \sim \frac{e^2}{4\pi\epsilon_0\hbar c} \sim \frac{1}{K_{\rm cond}(\mu)\,\text{(conv. factors)}}.
\end{equation}
The run $\alpha$ corresponds to the change in stiffness $K_{\rm cond}$ with energy/scale — physically: at higher energies the condensate "softens" (smaller $K$), which increases $\alpha$.

\subsection{Fine structure α ≈ 1/137}

\paragraph{Microscopic origin.}
The fine structure constant should be derived from the ratio:
\begin{equation}
\alpha_{\rm em} \sim \frac{(\text{topological vortex energy})}{(\text{condensate energy})} \sim \frac{e^2/R_{\rm proj}}{\rho_{\rm ent}\,V_{\rm proj}}.
\end{equation}
Numerically (estimated):
\begin{equation}
\alpha_{\rm em} \sim \frac{(1.6\times 10^{-19}\,{\rm C})^2/(4\pi\epsilon_0\cdot 2.6\times 10^{-2}\,{\rm m})}{(6\times 10^{-9}\,{\rm kg/m}^3)(7.2\times 10^{-5}\,{\rm m}^3)(c^2)} \sim ?
\end{equation}
This derivation requires a precise specification of the vortex (radius, core energy). It is a subject of ongoing research.

\paragraph{EFT approach.}
The preprint takes $\alpha\approx 1/137$ as input and models its behavior using NP--RG. A microscopic derivation should provide a value from first principles.

\subsection{Mapping Summary}

\begin{table}[h]
\centering
\caption{Parameter Conversion Table.}
\begin{tabular}{llll}
\toprule
\textbf{Microscopic} & \textbf{EFT (preprint)} & \textbf{Relation} & \textbf{Note} \\
\midrule
$n_\nu$ & — & input & C$\nu$B density \\
$E_{\rm pair}$ & DAR gain & $\sim 10^{20}m_\nu$ & neutrino confinement \\
$\rho_{\rm ent}$ (micro) & $\rho_{\rm ent}$ (EFT) & identical & with $E_{\rm pair}$ factor \\
$V_{\rm proj}$ & — & derived & coarse-graining scale \\
$\alpha_G$ & $\kappa_{\rm grav}$ (invert) & \eqref{eq:G_from_micro} vs lagrangian & $\sim 1$ \\
$g$ (self-int.) & $\lambda$ (quartic) & $\lambda\sim g/V_{\rm proj}^2$ & $\lambda\sim 10^{-31}$ \\
$K_{\rm cond}$ & run $\alpha(\mu)$ & \eqref{eq:c_stiffness} & stiffness $\sim 10^8$ Pa \\
$\Lambda_{\rm micro}$ & $\Lambda_{\rm QCT}$ & $\Lambda_{\rm QCT}\ll\Lambda_{\rm micro}$ & EFT cutoff vs micro \\
$\partial_\mu\theta$ & $A_\mu$ (photon) & Goldstone mode & gauge field \\
Poplar. vortex & charge $q=ne$ & quantization & $q\sim n\oint\nabla\theta$ \\
\bottomrule
\end{tabular}
\end{table}

\subsection{Practical fitting procedure}

\begin{enumerate}
\item \textbf{Input:} Fix $n_\nu=336\,{\rm cm}^{-3}$, $m_\nu\approx 0.1\,{\rm eV}$, $F_{\rm proj}=2.43\times 10^4$.
\item \textbf{Derivation of fundamental scales:} $V_{\rm proj}$, $R_{\rm proj}$ from definition.
\item \textbf{Fit $\alpha_G$ and $E_{\rm pair}$:} Replicate $G_{\rm eff}=6.67\times 10^{-11}$ using \eqref{eq:G_from_micro}.
\item \textbf{Consistency with EFT:} Verify that $\kappa_{\rm grav}\sim 10^{-37}$ matches microscopic values.
\item \textbf{Self-interaction:} Estimate $g$ from post-Newtonian corrections (perihelion), convert to $\lambda$.
\item \textbf{Stiffness $K$:} Calculate from \eqref{eq:c_stiffness}, compare with RG during $\alpha$.
\item \textbf{Testable predictions:} Variation of $G(z)$ with redshift, dispersion of light in gravitational fields, changes in atomic spectra.
\end{enumerate}

\vspace{0.5cm}
\noindent\emph{This mapping table allows a direct connection of microscopic calculations with phenomenological fits of the EFT preprint.}
