% Kosmologické korekce k mikroskopickému odvození
% Časová evoluce E_pair, horizont, slabá/silná interakce
% Datum: 2025-10-15

\section{Kosmologické korekce a časová evoluce}
\label{sec:cosmo_corrections}

V této sekci adresujeme kritickou mezeru v mikroskopickém odvození: \emph{časovou evoluci} parametrů kondenzátu od Velkého třesku a příspěvky od slabé a silné interakce. Ukazujeme, že numerické nesrovnalosti (faktor $\alpha\sim 10^{-10}$, příliš vysoká $\rho_{\rm eff}$) jsou důsledkem ignorování kosmologické historie.

\subsection{Časová evoluce vazbové energie E\_pair(t)}

\paragraph{Neutrinový confinement s kosmologickou expanzí.}
Původní odhad $E_{\rm pair}\sim 10^{20}\times m_\nu$ předpokládal \emph{statický} vesmír. Ve skutečnosti páry $\nu\bar\nu$ vznikly v raném vesmíru (při $T\sim 1\,{\rm MeV}$, $t\sim 1\,{\rm s}$) a od té doby jsou „natahované" expanzí.

\paragraph{Lineární confinement s Hubble parametrem.}
Analogicky k QCD string tension, navrhujeme:
\begin{equation}\label{eq:E_pair_evolution}
E_{\rm pair}(t) = E_0 + \kappa_{\rm conf}\,\int_0^t H(t')\,dt' = E_0 + \kappa_{\rm conf}\,\ln\!\left(\frac{a(t)}{a_0}\right),
\end{equation}
kde:
\begin{itemize}
  \item $H(t)$ je Hubbleův parametr,
  \item $a(t)$ je škálový faktor,
  \item $\kappa_{\rm conf}$ je konfinační konstanta (dimenze: energie).
\end{itemize}

\paragraph{Dnešní hodnota (t = 13.8 Gyr).}
Od nukleosyntézy ($t_{\rm BBN}\sim 3\,{\rm min}$, $a_{\rm BBN}/a_0\sim 10^{-9}$) do dnes:
\begin{equation}
E_{\rm pair}(t_0) = E_0 + \kappa_{\rm conf}\,\ln(10^9) \approx E_0 + 20\,\kappa_{\rm conf}.
\end{equation}

Pokud $E_0\sim m_\nu c^2$ a požadujeme $E_{\rm pair}(t_0)\sim 10^{20}\times m_\nu$:
\begin{equation}
\kappa_{\rm conf} \approx \frac{10^{20}\,m_\nu}{20} = 5\times 10^{18}\,m_\nu \sim 5\times 10^{17}\,{\rm eV}.
\end{equation}

\paragraph{Korekce k Λ\_QCT vztahu.}
Vztah $E_{\rm pair}=\Lambda_{\rm QCT}^2/m_\nu$ platí \emph{dnes}, ale historicky:
\begin{equation}
E_{\rm pair}(z) = E_0 + \kappa_{\rm conf}\,\ln(1+z).
\end{equation}
Pro $z\to\infty$ (raný vesmír): $E_{\rm pair}\to\infty$ logaritmicky, ne kvadraticky.

\textbf{Důsledek:} Skutečný vztah je:
\begin{equation}\label{eq:Lambda_time}
\boxed{\Lambda_{\rm QCT}(t) = \sqrt{E_{\rm pair}(t)\cdot m_\nu}}
\end{equation}
kde $\Lambda_{\rm QCT}$ \emph{běží} (mírně) s časem! Dnešní hodnota $\Lambda_{\rm QCT}(t_0)\approx 107\,{\rm TeV}$ je efektivní.

\subsection{Efektivní objem vesmíru a cutoff}

\paragraph{Kosmologický horizont.}
Projekční objemy $V_{\rm proj}$ nejsou izolované, ale vnořené do kosmologického objemu s horizontem:
\begin{equation}
R_{\rm horizon}(t) = c\,\int_0^t \frac{dt'}{a(t')} \approx \frac{c}{H_0} \sim 4.4\,{\rm Gpc}.
\end{equation}

Efektivní počet projekčních objemů v pozorovatelném vesmíru:
\begin{equation}
N_{\rm proj}^{\rm univ} = \frac{V_{\rm horizon}}{V_{\rm proj}} = \frac{4\pi R_{\rm horizon}^3/3}{V_{\rm proj}}.
\end{equation}

Numericky:
\begin{align}
V_{\rm horizon} &\approx \frac{4\pi}{3}(1.4\times 10^{26}\,{\rm m})^3 \approx 1.1\times 10^{79}\,{\rm m}^3,\\
N_{\rm proj}^{\rm univ} &= \frac{1.1\times 10^{79}}{7.2\times 10^{-5}} \approx 1.5\times 10^{83}.
\end{align}

\paragraph{Kolektivní redukce G\_eff.}
Gravitace je \emph{kolektivní jev} sumovaný přes všechny překryvy projekčních objemů. Pokud je frakce překrývání $f_{\rm overlap}\ll 1$, pak:
\begin{equation}
\alpha_{\rm eff} = \frac{N_{\rm overlap}}{N_{\rm proj}^{\rm univ}} = f_{\rm overlap}.
\end{equation}

Pro geometrii, kde každý $V_{\rm proj}$ se překrývá jen s nejbližšími sousedy (např. 6 v kubické mříži):
\begin{equation}
f_{\rm overlap} \sim \frac{6}{N_{\rm proj}^{\rm univ}} \sim \frac{6}{10^{83}} \sim 10^{-83}.
\end{equation}

❌ \textbf{To je příliš malé!} Musí existovat dlouhodosahový mechanismus.

\paragraph{Alternativa: Screening exponenciální.}
Projekční překryvy klesají exponenciálně s vzdáleností:
\begin{equation}
\alpha(r) = \alpha_0\,e^{-r/\lambda_{\rm screen}},
\end{equation}
kde $\lambda_{\rm screen}$ je screeningový radius. Pro $\lambda_{\rm screen}\sim 10\,R_{\rm proj}\approx 26\,{\rm cm}$:
\begin{equation}
\alpha_{\rm eff} = \alpha_0\int_0^\infty 4\pi r^2\,e^{-r/\lambda}\,dr = \alpha_0\cdot 8\pi\lambda^3 \sim \alpha_0\times 10^3.
\end{equation}

Pokud $\alpha_0\sim 1$ a požadujeme $\alpha_{\rm eff}\sim 10^{-10}$:
\begin{equation}
\lambda_{\rm screen} \sim R_{\rm proj}\times(10^{-13})^{1/3} \sim 0.05\,{\rm mm}.
\end{equation}

\textbf{Fyzikální interpretace:} Screening je \emph{velmi silný} na škále submilimetrů!

\subsection{Příspěvky slabé interakce}

\paragraph{W, Z bosony jako excitace kondenzátu.}
Kromě fotonu (Goldstone mód U(1)), mohou existovat těžší excitace odpovídající zlomení SU(2)$_L\times$U(1)$_Y\to$U(1)$_{\rm em}$. Navrhujeme:
\begin{align}
W^\pm &\sim \text{nabité excitace kondenzátu s } m_W\sim 80\,{\rm GeV},\\
Z^0 &\sim \text{neutrální excitace } m_Z\sim 91\,{\rm GeV}.
\end{align}

\paragraph{Příspěvek k ρ\_eff.}
Efektivní hustota zahrnuje všechny kanály:
\begin{equation}
\rho_{\rm eff}^{\rm total} = \rho_{\rm eff}^{(\gamma)} + \rho_{\rm eff}^{(W)} + \rho_{\rm eff}^{(Z)} + \rho_{\rm eff}^{(g)} + \cdots
\end{equation}

Pro slabé bosony (hmotné excitace):
\begin{equation}
\rho_{\rm eff}^{(W,Z)} = n_{\rm pairs}^{(W,Z)}\cdot m_{W,Z}\,c^2,
\end{equation}
kde $n_{\rm pairs}^{(W,Z)}$ je hustota virtuálních párů. V elektroslabeém vakuu ($T\ll m_W$):
\begin{equation}
n_{\rm pairs}^{(W,Z)} \sim n_\nu\times\exp\!\left(-\frac{m_W}{T_{\rm eff}}\right) \approx 0\quad(\text{Boltzmann potlačeno}).
\end{equation}

\textbf{Dnes:} Příspěvek je zanedbatelný ($T_{\rm CMB}\approx 2.7\,{\rm K}\ll m_W$).

\textbf{V raném vesmíru} ($T>m_W$): Slabé bosony byly v termální rovnováze a přispívaly významně k $\rho_{\rm eff}$. To ovlivnilo \emph{časovou evoluci} $G_{\rm eff}(t)$.

\subsection{Příspěvky silné interakce}

\paragraph{Gluony jako další excitace.}
Podobně jako fotony (U(1)) a W/Z (SU(2)), gluony mohou být octonická excitace odpovídající SU(3)$_c$. V kondenzátovém jazyce:
\begin{equation}
g_a^\mu \sim \text{SU(3) barevné excitace kondenzátu},\quad a=1,\ldots,8.
\end{equation}

\paragraph{Konfinement a dekonfinement.}
V konfinované fázi ($T<T_c\approx 170\,{\rm MeV}$) jsou gluony „uzavřené" v hadronech → jejich příspěvek k $\rho_{\rm eff}$ je absorbován do hadronické hustoty.

V dekonfinované fázi (quark-gluon plasma, raný vesmír $T>T_c$):
\begin{equation}
\rho_{\rm eff}^{(g)} = n_g\cdot\langle E_g\rangle \sim \frac{\pi^2}{30}g_g\,T^4,
\end{equation}
kde $g_g=16$ (8 gluonů × 2 helicity).

\paragraph{Vliv na G\_eff v raném vesmíru.}
Při $T\sim 1\,{\rm GeV}$ (před hadronizací):
\begin{equation}
\rho_{\rm eff}^{\rm total}(T=1\,{\rm GeV}) \sim n_\nu\,E_{\rm pair} + \rho_{\rm QGP},
\end{equation}
kde $\rho_{\rm QGP}\sim (1\,{\rm GeV})^4\sim 10^{36}\,{\rm eV}^4$.

Poměr:
\begin{equation}
\frac{\rho_{\rm QGP}}{\rho_{\rm eff}^{(\nu)}} \sim \frac{10^{36}}{n_\nu\,E_{\rm pair}} \sim \frac{10^{36}}{3\times 10^8\times 10^{29}} \sim 10^{-1}.
\end{equation}

✅ \textbf{Gluony přispívají ~10\% k efektivní hustotě při vysokých teplotách!}

\subsection{Rekalkulace G\_eff s korekcemi}

\paragraph{Upravený vzorec.}
Zahrneme všechny korekce:
\begin{equation}\label{eq:G_eff_corrected}
\boxed{G_{\rm eff}(t) = \alpha_{\rm geom}(t)\cdot f_{\rm screen}\cdot\frac{\rho_{\rm eff}^{\rm total}(t)\,V_{\rm proj}}{R_{\rm proj}\,M_{\rm Pl}^2}}
\end{equation}

kde:
\begin{itemize}
  \item $\alpha_{\rm geom}(t)$ zahrnuje časovou evoluci překryvů,
  \item $f_{\rm screen} = e^{-R_{\rm proj}/\lambda_{\rm screen}}$ je screening faktor,
  \item $\rho_{\rm eff}^{\rm total}(t) = \rho_{\rm eff}^{(\nu)}(t) + \rho_{\rm eff}^{(W,Z)}(t) + \rho_{\rm eff}^{(g)}(t)$.
\end{itemize}

\paragraph{Časová evoluce (schematicky).}
\begin{align}
t<10^{-12}\,{\rm s} &: \rho_{\rm eff}\sim T^4\gg\rho_\nu,\quad G_{\rm eff}\sim G_0\times(T/\Lambda_{\rm QCT})^4,\\
10^{-12}<t<10^{-6}\,{\rm s} &: \text{EW fázový přechod, }W/Z\text{ zmizí},\\
10^{-6}<t<1\,{\rm s} &: \text{QCD fázový přechod, gluony konfinují},\\
t>1\,{\rm s} &: \text{Pouze }\rho_{\rm eff}^{(\nu)},\quad G_{\rm eff}\to G_0.
\end{align}

\paragraph{Dnešní hodnota (t\_0 = 13.8 Gyr).}
Po všech fázových přechodech:
\begin{equation}
\rho_{\rm eff}^{\rm total}(t_0) \approx n_\nu(t_0)\cdot E_{\rm pair}(t_0),
\end{equation}
kde $E_{\rm pair}(t_0)\sim 10^{20}\times m_\nu$ (z logaritmického confinementu).

Ale: \textbf{screening faktor} $f_{\rm screen}\sim 10^{-10}$ redukuje efektivní coupling!

\subsection{Vyřešení numerické výzvy α ~ 10⁻¹⁰}

\paragraph{Klíčový insight.}
Faktor $\alpha\sim 10^{-10}$ není artefakt jednotek, ale \emph{fyzikální screening} na submilimetrových škálách. Mechanismus:

\begin{enumerate}[label=(\arabic*)]
  \item Projekční objemy $V_{\rm proj}\sim 72\,{\rm cm}^3$ jsou efektivní na škále $R_{\rm proj}\sim 2.6\,{\rm cm}$.
  \item Gravitační síla je kolektivní jev sumovaný přes překryvy.
  \item V hustém (baryonickém) prostředí dochází k \emph{dekoherenci} kondenzátu na kratších škálách.
  \item Efektivní screeningový radius: $\lambda_{\rm screen}\sim 0.05\,{\rm mm}$ (submilimetr!).
  \item Exponenciální potlačení: $f_{\rm screen}=e^{-26\,{\rm cm}/0.05\,{\rm mm}}\sim e^{-5200}\sim 10^{-2260}$.
\end{enumerate}

❌ \textbf{To je absurdně malé!} Musíme revidovat mechanismus.

\paragraph{Alternativní vysvětlení: Efektivní teorie s cutoffem.}
Místo exponenciálního screeningu, použijeme \emph{momentový cutoff}:
\begin{equation}
\alpha_{\rm eff} = \int_0^{\Lambda_{\rm IR}} \frac{d^3k}{(2\pi)^3}\,\alpha(k)\,e^{-k^2/\Lambda_{\rm QCT}^2},
\end{equation}
kde $\Lambda_{\rm IR}\sim 1/R_{\rm proj}$ je IR cutoff. Po integraci:
\begin{equation}
\alpha_{\rm eff} \sim \alpha_0\times\left(\frac{\Lambda_{\rm IR}}{\Lambda_{\rm QCT}}\right)^3 = \alpha_0\times\left(\frac{1/(2.6\,{\rm cm})}{107\,{\rm TeV}}\right)^3.
\end{equation}

Převod:
\begin{equation}
\frac{1}{2.6\,{\rm cm}} = \frac{\hbar c}{2.6\,{\rm cm}} \approx 7.6\times 10^{-6}\,{\rm eV}.
\end{equation}

\begin{equation}
\alpha_{\rm eff} \sim \left(\frac{7.6\times 10^{-6}}{1.45\times 10^{14}}\right)^3 \sim (5.2\times 10^{-20})^3 \sim 1.4\times 10^{-58}.
\end{equation}

❌ \textbf{Stále příliš malé!}

\subsection{Otevřená otázka: Správný screening mechanismus}

\paragraph{Závěr sekce.}
Časová evoluce ($E_{\rm pair}(t)$, slabá/silná interakce) \emph{zlepšuje} fyzikální interpretaci, ale numerický problém $\alpha\sim 10^{-10}$ přetrvává. Možné směry:

\begin{itemize}
  \item[\ding{51}] \textbf{Správné:} Kosmologická historie, příspěvky W/Z/gluonů.
  \item[\ding{55}] \textbf{Nedořešené:} Screening mechanismus (exponenciální? power-law? něco jiného?).
  \item[\ding{43}] \textbf{Hypotéza:} Gravitace není kolektivní suma, ale \emph{rezonanční jev} — překryvy musí být v koherentní fázi. Pokud je fázová koherence~$\sim 10^{-10}$, pak $\alpha_{\rm eff}\sim 10^{-10}$ je přirozený.
\end{itemize}

\vspace{0.5cm}
\noindent\hrulefill

\noindent\emph{Tato sekce identifikuje klíčové kosmologické korekce a otevírá program výzkumu screeningu a časové evoluce QCT parametrů. Numerické hodnoty vyžadují další zpřesnění.}
