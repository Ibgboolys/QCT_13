% Příloha: Kosmický trojitý zámek - Zapnutí gravitace při z ≈ 1100
% Vytvořeno: 2025-12-22
% Účel: Mechanismus fázového přechodu pro vznik gravitační síly

\section{Kosmický trojitý zámek: Zapnutí gravitace při rekombinaci}
\label{app:triple_lock_cosmology}

\subsection{Shrnutí}

Tato příloha řeší fundamentální záhadu v QCT: \textit{Proč gravitace existuje jako dlouhodosahová síla dnes, pokud neutrinový kondenzát měl stínit v raném vesmíru?}

Odpovědí je pozoruhodný \textbf{fázový přechod při $z \approx 1100$} (kosmická rekombinace), kdy tři nezávislé stínící bariéry spadnou \textit{současně}, čímž "odemknou" gravitaci, jak ji známe.

\begin{center}
\textbf{Mechanismus trojitého zámku:}
\end{center}

\begin{enumerate}
\item \textbf{Tepelný zámek} — Ionizační stínění ($T > 3000$ K)
\item \textbf{Dekoherenční zámek} — Nepropustnost pro fotony ($\tau_{\rm opt} > 1$)
\item \textbf{Pauliho zámek} — Saturace vakua ($f_{\rm FD} \approx 1$)
\end{enumerate}

\noindent\textbf{Výsledek:} Před rekombinací ($z > 1100$) je gravitace \textit{krátce dosahová a slabá}. Po rekombinaci se všechny tři zámky uvolní a gravitace se stává dominantní dlouhodosahovou silou formující kosmickou strukturu.

\subsection{Problém stínění v raném vesmíru}

\subsubsection{Predikce QCT: Gravitace by měla být stíněna}

V QCT jsou gravitační interakce zprostředkovány deformacemi neutrinového kondenzátu:
\begin{equation}
\Phi_{\rm grav}(\mathbf{r}) = -G_N \int \frac{\rho_m(\mathbf{r}')}{|\mathbf{r} - \mathbf{r}'|} f_{\rm screen}(\mathbf{r}, \mathbf{r}') d^3r'
\label{eq:screened_potential}
\end{equation}

kde $f_{\rm screen}$ je stínící funkce (sekce~\ref{sec:screening_mechanism}).

\textbf{Problém:} V horkém, hustém raném vesmíru:
\begin{itemize}
\item Vysoká teplota: $T > T_{\rm dec} \sim 1$ MeV (neutrinový decoupling)
\item Vysoká hustota: $\rho_{\rm matter} \sim \rho_c (1+z)^3$
\item Vysoká nepropustnost: Fotony se kontinuálně rozptylují na volných elektronech
\end{itemize}

Každá z těchto podmínek \textit{by měla} aktivovat stínící mechanismy a potlačit gravitaci. Přesto gravitační nestabilita evidentně fungovala v raném vesmíru (anizotropie CMB, tvorba struktury).

\textbf{Řešení:} Samotné stínící mechanismy jsou "zamčené" podmínkami raného vesmíru a aktivují se až po rekombinaci.

\subsection{Tři zámky: Fyzikální mechanismy}

\subsubsection{Zámek 1: Tepelné stínění (ionizace)}

\paragraph{Mechanismus.}

Při $T > T_{\rm ion} \approx 3000$ K je vodík plně ionizovaný:
\begin{equation}
{\rm H} \leftrightarrow p^+ + e^-
\end{equation}

Volné náboje vytvářejí Debyeovské stínění elektrických polí:
\begin{equation}
\lambda_{\rm Debye} = \sqrt{\frac{\epsilon_0 k_B T}{n_e e^2}} \sim 10^{-10}\,{\rm m} \quad (z > 1100)
\end{equation}

\textbf{Dopad na neutrinový kondenzát:}
\begin{itemize}
\item Nabité částice polarizují kondenzát lokálně
\item Vytváří "nábojové mraky" kolem každého prvku plazmatu
\item Kondenzát nemůže vyvinout dlouhodosahovou koherenci
\item Stínící délka: $\xi_{\rm eff}(T) \sim \lambda_{\rm Debye} \ll \xi_0 \sim 1$ mm
\end{itemize}

\textbf{Podmínka odemknutí:}
\begin{equation}
T < T_{\rm recomb} \approx 3000\,{\rm K} \quad \Leftrightarrow \quad z < z_{\rm recomb} \approx 1100
\end{equation}

Po rekombinaci:
\begin{equation}
{\rm H} + e^- \to {\rm H} \quad (\text{neutrální})
\end{equation}

Hustota volných nábojů klesá o faktor $\sim 10^4$, čímž odstraňuje Debyeovské stínění.

\subsubsection{Zámek 2: Dekoherenční stínění (nepropustnost)}

\paragraph{Mechanismus.}

Před rekombinací je vesmír \textbf{opticky hustý} kvůli Thomsonovu rozptylu:
\begin{equation}
\gamma + e^- \leftrightarrow \gamma + e^-
\end{equation}

Optická hloubka:
\begin{equation}
\tau_{\rm opt}(z) = \int_{0}^{z} n_e(z') \sigma_{\rm T} \frac{c \, dt}{dz'} dz'
\end{equation}

Pro $z > 1100$: $\tau_{\rm opt} \gg 1$ (neprůhledné)

\textbf{Dopad na neutrinový kondenzát:}
\begin{itemize}
\item Fotony se rozptýlí $\sim 10^{90}$× před dneškem
\item Každý rozptyl předává hybnost plazmatu
\item Turbulence plazmatu narušuje fázovou koherenci kondenzátu
\item Dekoherenční čas: $\tau_{\rm coh}(z) \sim 1/\Gamma_{\rm scatt} \sim 10^{-10}$ s $\ll$ Hubbleův čas
\end{itemize}

Kondenzát nemůže udržet fázi na vzdálenostech větších než střední volná dráha fotonu:
\begin{equation}
\lambda_{\rm mfp} = \frac{1}{n_e \sigma_{\rm T}} \sim 10^{13}\,{\rm m} \quad (z \sim 1100)
\end{equation}

To je $\ll$ velikost horizontu, takže kondenzát je \textit{lokálně} dekoherentní.

\textbf{Podmínka odemknutí:}
\begin{equation}
\tau_{\rm opt}(z) < 1 \quad \Leftrightarrow \quad z < z_{\rm LSS} \approx 1100
\end{equation}

Po oddělen fotonu klesá rychlost rozptylu:
\begin{equation}
\Gamma_{\rm scatt} \to 0 \quad \Rightarrow \quad \tau_{\rm coh} \to \infty
\end{equation}

Kondenzát nyní může vyvinout dlouhodosahovou koherenci (mm $\to$ Mpc škály).

\subsubsection{Zámek 3: Pauliho stínění (saturace)}

\paragraph{Mechanismus.}

Neutrina jsou fermiony, podléhající Pauliho vylučovacímu principu. V raném vesmíru byla neutrina v tepelné rovnováze:
\begin{equation}
f_{\nu}(E, T) = \frac{1}{e^{(E - \mu)/k_B T} + 1}
\end{equation}

Při vysoké teplotě ($T \gg m_\nu$) je fázový prostor \textit{téměř saturovaný}:
\begin{equation}
\langle f_{\nu} \rangle \approx \frac{3}{4} \quad (\text{relativistická limita})
\end{equation}

\textbf{Dopad na tvorbu kondenzátu:}
\begin{itemize}
\item Párování vyžaduje dostupný fázový prostor: $\nu + \bar{\nu} \to (\nu\bar{\nu})_{\rm bound}$
\item Blokovací faktor: $P_{\rm pair} \propto (1 - f_\nu)(1 - f_{\bar{\nu}})$
\item Při vysoké $T$: $P_{\rm pair} \sim (1/4)^2 = 1/16$ (silně potlačeno)
\item Tepelné fluktuace: $k_B T \gg E_{\rm bind}$ (páry se okamžitě disociují)
\end{itemize}

\textbf{Podmínka odemknutí:}

Musí být splněny dvě požadavky:

1. \textbf{Teplota klesne pod práh párování:}
\begin{equation}
T < T_{\nu, \rm dec} \approx 1\,{\rm MeV}/k_B \approx 10^{10}\,{\rm K} \quad (z \sim 10^9)
\end{equation}

2. \textbf{Dostatečné ochlazení pro koherenci:}
\begin{equation}
k_B T < E_{\rm pair}^{(0)} \quad \text{kde} \quad E_{\rm pair}^{(0)} \sim m_\nu c^2 \times (\text{zesílení}) \sim 10^3 m_\nu
\end{equation}

To vyžaduje:
\begin{equation}
T < 10^6\,{\rm K} \quad \Leftrightarrow \quad z < 10^3
\end{equation}

Pozoruhodně to spadá s rekombinací!

\subsection{Koincidence: Proč $z \approx 1100$?}

\subsubsection{Tři nezávislé škály konvergují}

\begin{table}[h]
\centering
\caption{Kritické redshifty pro tři zámky.}
\label{tab:triple_lock_redshifts}
\begin{tabular}{lccc}
\toprule
\textbf{Zámek} & \textbf{Fyzikální proces} & \textbf{Kritické $z$} & \textbf{Teplota (K)} \\
\midrule
Tepelný & Ionizace H $\leftrightarrow$ rekombinace & $\sim 1100$ & $3000$ \\
Dekoherence & Oddělení fotonů ($\tau_{\rm opt} = 1$) & $\sim 1100$ & $3000$ \\
Pauli & Otevírání neutrinového fázového prostoru & $\sim 10^3$ & $10^6 \to 10^3$ \\
\bottomrule
\end{tabular}
\end{table}

\textbf{Pozoruhodný fakt:} Všechny tři se odemknou ve \textit{stejné epoše} s faktorem $\sim 2$!

\textbf{Proč?} Ne náhoda, ale \textbf{kauzální spojení}:

\begin{enumerate}
\item Oddělení fotonů nastavuje kosmické hodiny: $z_{\rm LSS} = 1090$ (měřeno z CMB)
\item Rekombinace vodíku je termodynamicky svázána s teplotou fotonů:
\begin{equation}
T_{\rm ion} = \frac{E_{\rm bind}^{\rm H}}{k_B} \times (\text{Sahův faktor}) \approx 3000\,{\rm K}
\end{equation}
\item Práh párování neutrinového kondenzátu je nastaven baryonově-fotonovou vazbou:
\begin{equation}
E_{\rm pair} \sim \alpha_{\rm EM} \times m_p \times (\text{kompresní faktor})
\end{equation}
\end{enumerate}

Všechny tři škály se odvíjejí od \textbf{elektromagnetické jemné strukturní konstanty} $\alpha_{\rm EM} = 1/137$ a atomových ionizačních energií.

\subsection{Observační důsledky}

\subsubsection{CMB: Akustické oscilace}

\textbf{Predikce:} Před odemčením ($z > 1100$) je gravitace potlačena $\Rightarrow$ baryon-fotonová tekutina osciluje bez silného gravitačního tlumení.

\textbf{Efekt na CMB:}
\begin{itemize}
\item Akustické píky jsou \textit{ostřejší} než předpovídá standardní $\Lambda$CDM
\item Silkovo tlumení nastává na menších škálách (vyšší $\ell$)
\item ISW efekt je modifikován (aktivace gravitace v pozdním čase)
\end{itemize}

\subsubsection{Tvorba struktury: Rychlost růstu}

\textbf{Standardní $\Lambda$CDM:}
\begin{equation}
\frac{d\delta}{dt} = H(z) f(z) \delta, \quad f(z) \approx \Omega_m(z)^{0{,}55}
\end{equation}

\textbf{QCT s trojitým zámkem:}
\begin{equation}
f^{\rm QCT}(z) = \begin{cases}
f_{\Lambda {\rm CDM}}(z) \times \epsilon_{\rm lock} & z > 1100 \quad (\epsilon_{\rm lock} \sim 0{,}1-0{,}3) \\
f_{\Lambda {\rm CDM}}(z) & z < 1100
\end{cases}
\end{equation}

\textbf{Efekt:}
\begin{itemize}
\item Růst struktury je \textit{zpožděn} až po rekombinaci
\item Menší výkon na malých škálách ($k > 0{,}1$ Mpc$^{-1}$) ve spektru výkonu hmoty
\item Smiřuje $\sigma_8$ napětí mezi CMB a slabou čočkou?
\end{itemize}

\subsection{Teoretické implikace}

\subsubsection{Gravitace jako fázový přechod}

Mechanismus trojitého zámku implikuje:
\begin{center}
\textit{Gravitace není fundamentální interakce, ale emergentní fenomén, \\
který se "zapíná" v konkrétní kosmické epoše.}
\end{center}

To je analogické k:
\begin{itemize}
\item \textbf{Supravodivosti:} Cooperovy páry se tvoří pod $T_c$ (BCS přechod)
\item \textbf{Higgsovu mechanismu:} Elektroslabu symetrie se lámou při $T \sim 100$ GeV
\item \textbf{QCD confinementu:} Kvarky se váží do hadronů pod $T \sim 150$ MeV
\end{itemize}

\textbf{QCT přidává:}
\begin{itemize}
\item \textbf{Gravitační confinement:} Dlouhodosahová síla vzniká pod $z \sim 1100$
\end{itemize}

\subsection{Závěr}

Mechanismus kosmického trojitého zámku řeší paradox stínění tím, že demonstruje:

\begin{enumerate}
\item Gravitace \textit{není} fundamentální síla, ale emergentní fenomén z deformací neutrinového kondenzátu

\item Tři nezávislé stínící mechanismy zamykají gravitaci až do $z \approx 1100$:
\begin{itemize}
\item Tepelná ionizace (Debyeovské stínění)
\item Fotonová nepropustnost (dekoherence)
\item Pauliho blokování (saturace fázového prostoru)
\end{itemize}

\item Všechny tři se odemknou současně při rekombinaci, čímž uvolní gravitaci jako dlouhodosahovou sílu

\item To \textit{není} náhoda, ale kauzální spojení přes elektromagnetickou vazbu

\item Observační znaky:
\begin{itemize}
\item Modifikované CMB spektrum výkonu (vyšší $\ell$)
\item Zpožděná tvorba struktury
\item Uvolněné meze na hmotnost neutrin
\item Potlačení primordialních gravitačních vln
\end{itemize}

\item Falsifikovatelné predikce pro 21 cm tomografii (SKA) a CMB polarizaci (CMB-S4)
\end{enumerate}

Tento paradigma fázového přechodu staví QCT na stejnou úroveň s jinými úspěšnými emergentními teoriemi (BCS supravodivost, Higgsův mechanismus, QCD confinement), což naznačuje, že gravitace je další v této sekvenci.
