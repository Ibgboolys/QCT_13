% ============================================================================
% Monografie: Teorie kvantové komprese (QCT)
% Připraveno pro: Nakladatelství Masarykovy univerzity (Munipress)
% Format: Camera-ready PDF
% Autoři: Boleslav Plhák, Marek Novák
% Rok: 2025
% ============================================================================

\documentclass[12pt,a4paper,twoside,openright]{book}

% ============================================================================
% KÓDOVÁNÍ A JAZYK
% ============================================================================
\usepackage[utf8]{inputenc}
\usepackage[T1]{fontenc}
\usepackage[czech,english]{babel}  % hlavní jazyk je czech
\usepackage{csquotes}  % České uvozovky

% ============================================================================
% FONTY
% ============================================================================
\usepackage{times}  % Times New Roman (podle požadavků Munipress)
\usepackage{mathptmx}  % Times pro matematiku

% ============================================================================
% ROZMĚRY STRÁNKY A OKRAJE
% ============================================================================
% Pro camera-ready PDF - profesionální knihové okraje
\usepackage[
    a4paper,
    inner=3cm,      % vnitřní okraj (hřbet) - širší pro vazbu
    outer=2.5cm,    % vnější okraj
    top=3cm,        % horní okraj
    bottom=3cm,     % dolní okraj
    headheight=15pt,
    headsep=1cm
]{geometry}

% ============================================================================
% MATEMATICKÉ BALÍČKY
% ============================================================================
\usepackage{amsmath}
\usepackage{amssymb}
\usepackage{amsthm}
\usepackage{physics}  % pro fyzikální notaci
\usepackage{siunitx}  % pro jednotky SI

% Nastavení siunitx pro češtinu
\sisetup{
    locale = DE,  % německý styl (čárka jako oddělovač desetinných míst)
    output-decimal-marker = {,},
    inter-unit-product = \ensuremath{{}\cdot{}},
    per-mode = symbol
}

% ============================================================================
% GRAFIKA A TABULKY
% ============================================================================
\usepackage{graphicx}
\usepackage{booktabs}  % profesionální tabulky
\usepackage{array}
\usepackage{multirow}
\usepackage{longtable}  % tabulky přes více stránek
\usepackage{xcolor}

% Cesta k obrázkům
\graphicspath{{../results/figures/}{./figures/}}

% ============================================================================
% ODKAZY A REFERENCE
% ============================================================================
\usepackage[
    colorlinks=true,
    linkcolor=black,
    citecolor=blue,
    urlcolor=blue,
    bookmarks=true,
    bookmarksnumbered=true,
    unicode=true
]{hyperref}

\usepackage[nameinlink]{cleveref}  % inteligentní křížové odkazy

% České názvy pro cleveref
\crefname{chapter}{kapitola}{kapitoly}
\Crefname{chapter}{Kapitola}{Kapitoly}
\crefname{section}{sekce}{sekce}
\Crefname{section}{Sekce}{Sekce}
\crefname{equation}{rovnice}{rovnice}
\Crefname{equation}{Rovnice}{Rovnice}
\crefname{figure}{obrázek}{obrázky}
\Crefname{figure}{Obrázek}{Obrázky}
\crefname{table}{tabulka}{tabulky}
\Crefname{table}{Tabulka}{Tabulky}

% ============================================================================
% BIBLIOGRAFIE (ČSN ISO 690)
% ============================================================================
\usepackage[
    backend=biber,
    style=iso-authoryear,  % styl dle ČSN ISO 690
    sortlocale=cs_CZ,
    autolang=other,
    bibencoding=UTF8
]{biblatex}

\addbibresource{references.bib}  % hlavní soubor s literaturou

% ============================================================================
% ZÁHLAVÍ A ZÁPATÍ
% ============================================================================
\usepackage{fancyhdr}
\pagestyle{fancy}
\fancyhf{}
\fancyhead[LE]{\leftmark}   % levé stránky: název kapitoly
\fancyhead[RO]{\rightmark}  % pravé stránky: název sekce
\fancyfoot[C]{\thepage}     % číslo stránky ve středu zápatí

\renewcommand{\headrulewidth}{0.4pt}
\renewcommand{\footrulewidth}{0pt}

% Styl pro kapitoly (bez záhlaví)
\fancypagestyle{plain}{
    \fancyhf{}
    \fancyfoot[C]{\thepage}
    \renewcommand{\headrulewidth}{0pt}
}

% ============================================================================
% ŘÁDKOVÁNÍ
% ============================================================================
\usepackage{setspace}
\onehalfspacing  % řádkování 1.5 (lze změnit na \singlespacing pro finální verzi)

% ============================================================================
% POZNÁMKY POD ČAROU
% ============================================================================
\usepackage[bottom]{footmisc}  % poznámky vždy na spodku stránky

% ============================================================================
% REJSTŘÍKY
% ============================================================================
\usepackage{makeidx}
\makeindex

% ============================================================================
% DALŠÍ UŽITEČNÉ BALÍČKY
% ============================================================================
\usepackage{enumitem}  % lepší seznamy
\usepackage{caption}   % lepší popisky obrázků a tabulek
\usepackage{subcaption}  % podobrázky

% Nastavení popisků
\captionsetup{
    font=small,
    labelfont=bf,
    format=hang,
    justification=justified,
    singlelinecheck=false
}

% ============================================================================
% VLASTNÍ PŘÍKAZY A DEFINICE
% ============================================================================

% Fyzikální konstanty
\newcommand{\hbar}{\ensuremath{\hslash}}
\newcommand{\Geff}{\ensuremath{G_{\mathrm{eff}}}}
\newcommand{\Epair}{\ensuremath{E_{\mathrm{pair}}}}
\newcommand{\LambdaQCT}{\ensuremath{\Lambda_{\mathrm{QCT}}}}
\newcommand{\Rproj}{\ensuremath{R_{\mathrm{proj}}}}
\newcommand{\fscreen}{\ensuremath{f_{\mathrm{screen}}}}

% Pole a operátory
\newcommand{\Psi}{\ensuremath{\Psi}}
\newcommand{\psinn}{\ensuremath{\Psi_{\nu\nu}}}

% Prostředí pro definice, věty, atd.
\theoremstyle{definition}
\newtheorem{definition}{Definice}[chapter]
\newtheorem{theorem}{Věta}[chapter]
\newtheorem{lemma}{Lemma}[chapter]
\newtheorem{corollary}{Důsledek}[chapter]

\theoremstyle{remark}
\newtheorem{remark}{Poznámka}[chapter]
\newtheorem{example}{Příklad}[chapter]

% ============================================================================
% METADATA
% ============================================================================
\title{
    {\Huge\bfseries Teorie kvantové komprese}\\[0.5cm]
    {\Large Mikroskopické odvození emergentní gravitace\\
    z neutrinového kondenzátu}
}

\author{
    Boleslav Plhák\\
    {\small ORCID: 0009-0003-7469-5212}\\[0.3cm]
    Marek Novák\\
    {\small ORCID: 0009-0008-2525-0109}\\[0.5cm]
    {\small\itshape Nezávislí badatelé, Znojmo, Česká republika}
}

\date{2025}

% ============================================================================
% ZAČÁTEK DOKUMENTU
% ============================================================================
\begin{document}

% ----------------------------------------------------------------------------
% PŘEDNÍ STRANA (frontmatter) - římské číslování
% ----------------------------------------------------------------------------
\frontmatter

% Titulní strana
\maketitle

% Autorská práva a informace o publikaci
\thispagestyle{empty}
\vspace*{\fill}
\noindent
\textbf{Teorie kvantové komprese: Mikroskopické odvození emergentní gravitace z~neutrinového kondenzátu}

\vspace{0.5cm}
\noindent
Autoři: Boleslav Plhák, Marek Novák\\
Recenzenti: [doplní nakladatelství]

\vspace{0.5cm}
\noindent
\textcopyright{} 2025 Boleslav Plhák\\
Tato teoretická práce je dostupná pod licencí\\
Creative Commons Attribution 4.0 International License.

\vspace{0.5cm}
\noindent
Zdrojový kód a výpočetní skripty jsou dostupné pod licencí MIT.

\vspace{0.5cm}
\noindent
DOI: \href{https://doi.org/10.5281/zenodo.17081478}{10.5281/zenodo.17081478}\\
GitHub: \url{https://github.com/Ibgboolys/QCT_13}

\vspace{0.5cm}
\noindent
Vydalo: Nakladatelství Masarykovy univerzity\\
Rybkova 19, 602\,00 Brno\\
\url{www.press.muni.cz}

\vspace{0.3cm}
\noindent
1. vydání, 2025\\
ISBN: [doplní nakladatelství]

\vspace{0.5cm}
\noindent
Sazba: [camera-ready od autorů]

\clearpage

% Obsah
\tableofcontents
\clearpage

% Předmluva
\chapter*{Předmluva}
\addcontentsline{toc}{chapter}{Předmluva}

[BUDE DOPLNĚNO - Předmluva obsahující:]
\begin{itemize}
    \item Motivaci k výzkumu teorie kvantové komprese
    \item Kontext současné teoretické fyziky a problém kvantové gravitace
    \item Stručný přehled struktury knihy
    \item Poděkování spolupracovníkům, recenzentům a institucím
\end{itemize}

\clearpage

% ----------------------------------------------------------------------------
% HLAVNÍ TEXT (mainmatter) - arabské číslování
% ----------------------------------------------------------------------------
\mainmatter

% ============================================================================
% ÚVOD
% ============================================================================
\chapter{Úvod}
\label{chap:uvod}

[BUDE DOPLNĚNO - Úvodní kapitola obsahující:]

\section{Problém emergentní gravitace}
\label{sec:problem-emergentni-gravitace}

[Text o současném stavu poznání v kvantové gravitaci, motivace pro emergentní přístupy]

\section{Cíle monografie}
\label{sec:cile}

Tato monografie představuje kompletní výklad \textbf{Teorie kvantové komprese} (Quantum Compression Theory, QCT), která navrhuje, že gravitace a elektromagnetismus vznikají jako kolektivní jevy z kondenzátu kosmického neutrinového pozadí.

Hlavní cíle práce jsou:
\begin{enumerate}
    \item Odvodit Einsteinovy rovnice z mikroskopického popisu zapletených neutrinových párů
    \item Vysvětlit slabost gravitace pomocí fundamentálního poměru hmotností $\fscreen = m_\nu/m_p \approx 10^{-10}$
    \item Předložit testovatelné predikce pro experimentální verifikaci
    \item Poskytnout konzistentní popis kosmologické evoluce parametrů teorie
\end{enumerate}

\section{Přehled metodologie}
\label{sec:metodologie}

[Popis metodologického přístupu - efektivní teorie pole, analogová gravitace, kosmologická fyzika]

\section{Struktura knihy}
\label{sec:struktura}

Monografie je strukturována následovně:

\textbf{\Cref{chap:zaklady}} zavádí základní pojmy teorie kvantové komprese, neutrinový kondenzát jako fundamentální pole a Gross-Pitaevskiiho popis.

\textbf{\Cref{chap:einstein}} odvozuje Einsteinovy rovnice obecné relativity z emergentní prostoročasové geometrie neutrinového kondenzátu.

[pokračování struktury...]

% ============================================================================
% KAPITOLA 1: Základy teorie kvantové komprese
% ============================================================================
\chapter{Základy teorie kvantové komprese}
\label{chap:zaklady}

[BUDE DOPLNĚNO]

\section{Neutrinový kondenzát jako fundamentální pole}
\label{sec:neutrino-kondenzat}

\section{Gross-Pitaevskiiho popis}
\label{sec:gross-pitaevskii}

\section{Základní předpoklady a omezení}
\label{sec:predpoklady}

% ============================================================================
% KAPITOLA 2: Odvození Einsteinových rovnic
% ============================================================================
\chapter{Odvození Einsteinových rovnic}
\label{chap:einstein}

[BUDE DOPLNĚNO]

\section{Emergentní prostoročasová geometrie}
\label{sec:emergentni-geometrie}

\section{Gravitační konstanta a fázová koherence}
\label{sec:gravitacni-konstanta}

\section{Submilimetrové stínění}
\label{sec:stineni}

% ============================================================================
% KAPITOLA 3-9: Další kapitoly
% ============================================================================

\chapter{Odvození Maxwellových rovnic}
\label{chap:maxwell}
[BUDE DOPLNĚNO]

\chapter{Mikroskopické odvození vazebné energie}
\label{chap:vazebna-energie}
[BUDE DOPLNĚNO]

\chapter{Efektivní teorie pole}
\label{chap:eft}
[BUDE DOPLNĚNO]

\chapter{Kosmologická evoluce parametrů}
\label{chap:kosmologie}
[BUDE DOPLNĚNO]

\chapter{Fenomenologie a testovatelné predikce}
\label{chap:fenomenologie}
[BUDE DOPLNĚNO]

\chapter{Temná energie z saturace kondenzátu}
\label{chap:temna-energie}
[BUDE DOPLNĚNO]

\chapter{Teoretické otázky}
\label{chap:teoreticke-otazky}
[BUDE DOPLNĚNO]

% ============================================================================
% ZÁVĚR
% ============================================================================
\chapter{Závěr}
\label{chap:zaver}

[BUDE DOPLNĚNO - Závěrečná kapitola obsahující:]
\begin{itemize}
    \item Shrnutí hlavních výsledků teorie kvantové komprese
    \item Zhodnocení testovatelných predikcí
    \item Budoucí výzkumné směry
    \item Implikace pro fundamentální fyziku
\end{itemize}

% ----------------------------------------------------------------------------
% ZADNÍ STRANA (backmatter)
% ----------------------------------------------------------------------------
\backmatter

% ============================================================================
% SUMMARY (anglicky)
% ============================================================================
\chapter*{Summary}
\addcontentsline{toc}{chapter}{Summary}

\textbf{Quantum Compression Theory: Microscopic Derivation of Emergent Gravity from Neutrino Condensate}

[BUDE DOPLNĚNO - Anglický souhrn celé monografie dle požadavků RIV]

% ============================================================================
% POZNÁMKY (pokud jsou použity v textu)
% ============================================================================
% \chapter*{Poznámky}
% \addcontentsline{toc}{chapter}{Poznámky}
% [Poznámky pod čarou jsou automaticky na každé stránce]

% ============================================================================
% SEZNAM LITERATURY
% ============================================================================
\printbibliography[title={Seznam použité literatury}]
\addcontentsline{toc}{chapter}{Seznam použité literatury}

% ============================================================================
% REJSTŘÍKY
% ============================================================================
\printindex
\addcontentsline{toc}{chapter}{Rejstřík}

% ============================================================================
% PŘÍLOHY (volitelně)
% ============================================================================
\appendix

\chapter{Matematické konstanty v QCT}
\label{app:mat-konstanty}
[BUDE DOPLNĚNO]

\chapter{Numerické výpočty}
\label{app:numericke-vypocty}
[BUDE DOPLNĚNO]

\end{document}
